\documentclass[11pt,letterpaper,oneside]{article}
\usepackage[utf8]{inputenc}
\usepackage[T1]{fontenc}
\usepackage{amssymb,amsmath,amsthm}
\usepackage{enumitem}
\usepackage[osf]{libertine}
\usepackage{carlito}
\usepackage{inconsolata}
\usepackage{sectsty}

\allsectionsfont{\sffamily\mdseries}
\usepackage[activate={true,nocompatibility},final,tracking=true,spacing=true]{microtype}
\microtypecontext{spacing=nonfrench}
%\usepackage{embrac}
\usepackage{typearea}   % Let the typearea package work out the best margins (wider than the defaults)
\usepackage{fancyhdr}
\pagestyle{fancy}
\lhead{}
\chead{}
\rhead{}
\lfoot{}
\cfoot{}
\rfoot{\thepage}
\renewcommand{\headrulewidth}{0pt}

\newcommand{\proofsep}{\vspace{-0.75em}}
\usepackage{booktabs}

\usepackage{graphicx}
% Improve math spacing around |,\left and \right
% See: https://tex.stackexchange.com/questions/2607/spacing-around-left-and-right/
\let\originalleft\left
\let\originalright\right
\renewcommand{\left}{\mathopen{}\mathclose\bgroup\originalleft}
\renewcommand{\right}{\aftergroup\egroup\originalright}

% Remove section numbering
\makeatletter
% we use \prefix@<level> only if it is defined
\renewcommand{\@seccntformat}[1]{%
    \ifcsname prefix@#1\endcsname
    \csname prefix@#1\endcsname
    \else
    \csname the#1\endcsname\quad
    \fi}
% define \prefix@section
\newcommand\prefix@section{}
\newcommand\prefix@subsection{}
\newcommand\prefix@subsubsection{}
\makeatother

% Define new commands for "such that", the blackboard-bold letters and "closure".
\newcommand{\st}{\textit{ s.t.\ }}
\newcommand{\R}{\ensuremath{\mathbb{R}}}
\newcommand{\N}{\ensuremath{\mathbb{N}}}
\newcommand{\Z}{\ensuremath{\mathbb{Z}}}
\newcommand{\Q}{\ensuremath{\mathbb{Q}}}
\newcommand{\cl}{\ensuremath{\text{cl\hspace{0.1em}}}}
\newcommand{\card}{\ensuremath{\text{card\hspace{0.1em}}}}

\DeclareMathOperator{\blackboardE}{\mathbb{E}}
\newcommand{\expected}[1]{\blackboardE\left[#1\right]}
\newcommand{\expectedwhen}[2]{\blackboardE_{#2}\left[#1\right]}
\newlength{\postparenlength}
\setlength{\postparenlength}{0.14em}
\newcommand{\postparen}{\hspace{\postparenlength}}
\usepackage{url}
\usepackage{xcolor}

\usepackage[biblatex]{embrac}
\usepackage{siunitx}
\sisetup{
	group-separator={,},
	group-digits = integer,
	input-ignore = {,},
	input-decimal-markers = {.}
	}
\usepackage[authordate,isbn=false,backend=biber,autopunct=true]{biblatex-chicago}
\DefineBibliographyExtras{american}{\stdpunctuation}
\renewcommand{\finalnamedelim}{\addspace \bibstring {and}\space}
%\renewcommand*{\bibfont}{\footnotesize}
%\DefineBibliographyStrings{english}{references = {References}}
%\setlength\bibitemsep{0pt}
\bibhang=\parindent
\addbibresource{./papers.bib}
\newcommand{\cex}{\textsc{cex}}
%\usepackage{footnote}
\usepackage{hyperref}
\hypersetup{colorlinks,
    linkcolor=black,
    filecolor=black,
    urlcolor=darkgray,
    citecolor=black,
    pdfpagemode=UseNone,
    pdftoolbar=false,
    pdftitle={Brief description},
    pdfauthor={Karl Dunkle Werner},
    pdfsubject={ARE second year paper},
    pdfcreator={},
    pdfproducer={},
    pdflang=en,
    unicode=true
}


\begin{document}


\part{One-sentence blurb}
How do the Alaska Permanent Fund payments affect consumers' behavior, particularly in the used car market?


\part{Question \& abstract (rough)}
\section{Consumption Effects of the Alaska Permanent Fund}
\noindent Residents of Alaska are eligible for large payments from the Alaska Permanent Fund, a government organization created to distribute oil revenues.
In 2015, 86\% of the population received the dividend, \$2072. (See table~\ref{table:APF-payment-summary} for details.)
If consumers deviate from perfect consumption smoothing, we might think that such a large payment would affect their purchase behavior.
Somewhat surprisingly, \textcite{hsieh2003} found that consumption was smoothed out in Consumer Expenditure Survey (\cex) data from 1980--2001.
I'm hoping to do two things in my second-year paper: extend Hsieh's analysis of consumers and expand the scope to firms.

Extending Hsieh's analysis is straightforward.
There's a decade more \cex{} data, including the recent financial crisis and substantial variation in the Permanent Fund payments, and it would be interesting to see if the original conclusions still hold.
Additionally, Hsieh's identification is a difference-in-differences between Alaska and the other 49 states.
I won't be able to fundamentally improve  % I have no qualms about splitting infinitives
on the underlying assumption of parallel trends, but I will be able to apply slightly newer methods, like synthetic controls.

Expanding the scope of analysis to firms is more interesting.
I have data on used wholesale car auctions from 2002 to 2014, where the buyers and sellers are auto dealerships and other players in the wholesale market.
In those data Alaskan firms buy vehicles from other states, presumably to sell in Alaska.
If consumers are actually smoothing their consumption, even of large purchases like cars, I would expect the Permanent Fund payments to have no effect on Alaskan wholesalers' behavior.
If I make the same diff-in-diff assumptions about wholesalers, I can test the hypothesis of consumption smoothing with the used car data, since wholesalers' auction behavior should be unchanged if their end-consumers' behavior has been smoothed out.

But I can go further.
If the consumption smoothing hypothesis didn't hold, I would expect auto wholesalers to smooth out consumers' behavior.
(But I wouldn't expect wholesalers to be totally unaffected by Permanent Fund timing, since it's costly to hold extra cars in inventory.)
That's to say, if consumers weren't smoothing their auto purchases, I'd expect to find that:
\begin{itemize}
    \item Alaskan wholesalers would increase their auction \emph{purchases} (and be willing to pay higher prices) in the days and weeks before the Permanent Fund payouts, bringing more cars to Alaska.
    \item Alaskan wholesalers would decrease their auction \emph{sales} (or require higher sale prices) in advance of the payouts, keeping cars in Alaska.
    \item Alaskan wholesalers would preferentially accumulate cars of the type that would be bought by people using their Permanent Fund dividends.%
    \footnote{I think this will be too noisy to measure, and it's not clear to me whether consumers would tend toward lower-price cars, since that's what they can afford, or higher-price cars, pooling together the dividend and their own savings to get a nicer used car.}
\end{itemize}
I need to think a bit more about the implications here.
Consumer spending and the permanent-income hypothesis are a big deal, and it probably matters how local firms act to smooth out consumers' behavior, but I'm not up enough on macro to know why.


There are, of course, a lot of caveats.
Alaska is different than other states.
For one thing, shipping cars there is probably expensive, and that cost probably varies by time of year, possibly violating the diff-in-diff assumptions.

I don't observe wholesalers' inventory or final prices, so if wholesalers (temporarily) draw down their inventory instead of buying in the auctions, I wouldn't be able to observe any behavior changes and it would look like consumers were smoothing their consumption, even if they weren't.
On the other hand, if wholesalers are credit constrained and unable to respond to consumers' changes in demand, I would again see little change in the auctions, regardless of consumers' smoothing behavior.

\noindent\begin{minipage}{\textwidth}
	\centering
%\begin{table}
%	\caption{Alaska Permanent Fund payment summary	\footnote{Numbers not adjusted for inflation. Accessed Sept.\ 16, 2016 from \url{http://pfd.alaska.gov/Division-Info/Summary-of-Applications-and-Payments}.}}
	\label{table:APF-payment-summary}
	\textsc{Table} \ref{table:APF-payment-summary}: Alaska Permanent Fund payment summary%
	\footnote{Numbers not adjusted for inflation. Accessed Sept.\ 16, 2016 from \url{http://pfd.alaska.gov/Division-Info/Summary-of-Applications-and-Payments}.}

	\vspace{1em}
    \begin{tabular}{SSSSSS}
& &  \multicolumn{2}{c}{Applications\hspace*{4em}}   &   &  \\
\multicolumn{1}{l}{Year} & \multicolumn{1}{l}{State pop.}          & \multicolumn{1}{l}{Received} & \multicolumn{1}{l}{Paid}  &  \multicolumn{1}{c}{Dividend (\$)}    &  \multicolumn{1}{r}{State total (mm\$)}      \\
\toprule

2015 & 737625 & 672741 & 637014 & 2072.00 & 1272.84134000 \\ 
2014 & 735601 & 670053 & 631306 & 1884.00 & 1189.28974800 \\ 
2013 & 736399 & 668362 & 631470 & 900.00 & 568.32300000 \\ 
2012 & 732298 & 673978 & 610633 & 878.00 & 536.135774 \\ 
2011 & 722190 & 672237 & 615122 & 1174.00 & 722.153228 \\ 
2010 & 710231 & 663938 & 611522 & 1281.00 & 783.359682 \\ 
2009 & 692314 & 654462 & 621146 & 1305.00 & 810.595530 \\ 
2008 & 679720 & 641291 & 610096 & 2069.00 & 1262.288624 \\ 
2007 & 676987 & 628895 & 595237 & 1654.00 & 984.521998 \\ 
2006 & 670053 & 623792 & 594029 & 1106.96 & 657.56634184 \\ 
2005 & 663253 & 627595 & 596936 & 845.76 & 504.86459136 \\ 
2004 & 656834 & 625535 & 599243 & 919.84 & 551.20768112 \\ 
2003 & 647747 & 619552 & 595571 & 1107.56 & 659.63061676 \\ 
2002 & 640544 & 612377 & 589420 & 1540.76 & 908.15475920 \\ 
2001 & 632241 & 608600 & 586230 & 1850.28 & 1084.68964440 \\ 
2000 & 627533 & 607910 & 583098 & 1963.86 & 1145.12283828 \\ 
1999 & 622000 & 589738 & 572877 & 1769.84 & 1013.90062968 \\ 
1998 & 617082 & 581803 & 565256 & 1540.88 & 870.99166528 \\ 
1997 & 609655 & 573057 & 554769 & 1296.54 & 719.28019926 \\ 
1996 & 605212 & 564362 & 546045 & 1130.68 & 617.40216060 \\ 
1995 & 601581 & 563020 & 541842 & 990.30 & 536.58613260 \\ 
1994 & 600622 & 557836 & 534599 & 983.90 & 525.99195610 \\ 
1993 & 596906 & 549066 & 527946 & 949.46 & 501.26360916 \\ 
1992 & 586722 & 542263 & 522636 & 915.84 & 478.65095424 \\ 
1991 & 569054 & 533692 & 512098 & 931.34 & 476.93735132 \\ 
1990 & 553171 & 531494 & 497608 & 952.63 & 474.03630904 \\ 
1989 & 538900 & 524272 & 507547 & 873.16 & 443.16973852 \\ 
1988 & 535000 & 532227 & 518150 & 826.93 & 428.47377950 \\ 
1987 & 541300 & 535578 & 529478 & 708.19 & 374.97102482 \\ 
1986 & 550700 & 540202 & 532294 & 556.26 & 296.09386044 \\ 
1985 & 543900 & 525145 & 518479 & 404.00 & 209.465516 \\ 
1984 & 524000 & 490413 & 481349 & 331.29 & 159.46611021 \\ 
1983 & 499100 & 465567 & 457209 & 386.15 & 176.55125535 \\ 
1982 & 464300 & 484344 & 469741 & 1000.00 & 469.741000 \\ 


\end{tabular}

%\end{table}
%\footnotetext{text}
\end{minipage}

\section{Data sources}

\begin{itemize}
    \item Permanent fund payment dates and amounts
    \item \cex
    \item Manheim auction data
    \item Equifax household debt, default, and origination? (zip--quarter data, 1991--2014)  (Access via Chicago Booth people)
    \item RL Polk new car sales (zip--month data, 1998--2014)  (Access via Chicago Booth people)
\end{itemize}
% References
\printbibliography
\end{document}
