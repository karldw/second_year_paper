\documentclass[11pt,letterpaper,oneside]{article}
\usepackage[utf8]{inputenc}
\usepackage[T1]{fontenc}
\usepackage{amssymb,amsmath,amsthm}
\allowdisplaybreaks
\usepackage{dsfont}
\usepackage[mathlf]{MinionPro}
\usepackage{carlito}
\usepackage{inconsolata}
\usepackage{sectsty}
\usepackage{setspace}
\usepackage[font=sf]{caption}
\allsectionsfont{\sffamily\mdseries}

\usepackage[activate={true,nocompatibility}, final, tracking=true, spacing=true, letterspace=80% only letterspace 80, rather than 100 (units are 1/1000 em)
]{microtype}
\microtypecontext{spacing=nonfrench}
\SetProtrusion{encoding={*},family={bch},series={*},size={6,7}}
              {1={ ,750},2={ ,500},3={ ,500},4={ ,500},5={ ,500},
               6={ ,500},7={ ,600},8={ ,500},9={ ,500},0={ ,500}}
\widowpenalty=10000
% \raggedbottom
% \usepackage[headsep=0pt]{typearea}
\usepackage[top=1.2in, bottom=1.4in]{geometry}
\usepackage{fancyhdr}
\pagestyle{fancy}
\lhead{}
\chead{}
\rhead{}
\lfoot{}
\cfoot{}
\rfoot{\thepage}
\renewcommand{\headrulewidth}{0pt}
% \setlength{\headheight}{10pt}

\newcommand{\proofsep}{\vspace{-0.75em}}
\usepackage{booktabs}

\usepackage{graphicx}
\usepackage{subcaption}
\usepackage{placeins}  % use \usepackage[section]{placeins} to automatically put a FloatBarrier before each section
% Improve math spacing around |,\left and \right
% See: https://tex.stackexchange.com/questions/2607/spacing-around-left-and-right/
\let\originalleft\left
\let\originalright\right
\renewcommand{\left}{\mathopen{}\mathclose\bgroup\originalleft}
\renewcommand{\right}{\aftergroup\egroup\originalright}

% \setlength{\parskip}{0.5\parindent plus 0.1\parindent minus 0.1\parindent}
% \setlength{\parskip}{\baselineskip}
\usepackage{url}
\usepackage{xcolor}

\usepackage[biblatex]{embrac}
\usepackage{hyphenat}
\doublehyphendemerits=20000  % Increase the penalty for multiple hyphenated lines in a row (note, in this case, >10000 isn't infinity)
\usepackage{dcolumn}
\usepackage{siunitx}
\sisetup{
	group-separator={,},
	group-digits = integer,
	input-ignore = {,()},
	input-decimal-markers = {.},
    round-mode=figures,
    round-precision=3,
    % input-symbols = ()
	}
\usepackage[authordate,isbn=false,backend=biber,autopunct=true,url=false]{biblatex-chicago}
% \DefineBibliographyExtras{american}{\stdpunctuation}
\renewcommand{\finalnamedelim}{\addspace \bibstring {and}\space}
%\renewcommand*{\bibfont}{\footnotesize}
%\DefineBibliographyStrings{english}{references = {References}}
%\setlength\bibitemsep{0pt}
\bibhang=\parindent
\addbibresource{./papers.bib}
\newcommand{\cex}{\textsc{cex}}
\newcommand{\apf}{\textsc{apf}}
\newcommand{\msrp}{\textsc{msrp}}
\newcommand{\eitc}{\textsc{eitc}}
\newcommand{\gdp}{\textsc{gdp}}
\newcommand{\vin}{\textsc{vin}}

\newcommand{\Var}{\text{Var}}


% Allow for non-justified columns in tables by using column types L and R
\usepackage{ragged2e}  % https://tex.stackexchange.com/a/7375
\newcolumntype{L}[1]{>{\RaggedRight\hspace{0pt}}p{#1}}
\newcolumntype{R}[1]{>{\RaggedLeft\hspace{0pt}}p{#1}}


\usepackage{gitinfo2}  % requires git hooks!  See the Code/git_hooks folder in this repo.
\newcommand{\snippet}[1]{\input{./Generated_snippets/#1}\hspace{-0.15em}}
%\usepackage{footnote}
\usepackage{hyperref}
\hypersetup{colorlinks,
    linkcolor=black,
    filecolor=black,
    urlcolor=darkgray,
    citecolor=black,
    pdfpagemode=UseNone,
    pdftoolbar=false,
    pdftitle={Cars in Alaska},
    pdfauthor={Karl Dunkle Werner},
    pdfsubject={ARE second year paper},
    pdfcreator={},
    pdfproducer={},
    pdflang=en,
    unicode=true
}


\graphicspath{{./Plots/}, {./Plots/Daily/}}



% Define new commands for "such that", the blackboard-bold letters and "closure".
\newcommand{\st}{\textit{ s.t.\ }}
\newcommand{\indicator}[1]{\mathds{1}\left\{#1\right\}}
\newcommand{\anticipation}{\indicator{t \in [W_0, W_1]}}
\newcommand{\isAlaska}{\indicator{s = \textsc{ak}}}
\newcommand{\postWindow}{\indicator{t > W_1}}
\newcommand{\ddEqn}[3]{%
\text{#1}_{\, sty} = & \  #2_1 \ \anticipation + #2_2 \ \isAlaska \times \anticipation + #2_3 \ \postWindow \\ \nonumber
&+ #2_4 \ \isAlaska \times \postWindow + #3 F_{ sty} + \varepsilon_{\, sty}
}
\newcommand{\ddEqnLog}[3]{%
\ln\left(\text{#1}_{\, sty}\right) = & \  #2_1 \ \anticipation + #2_2 \ \isAlaska \times \anticipation + #2_3 \ \postWindow \\ \nonumber
&+ #2_4 \ \isAlaska \times \postWindow + #3 F_{ sty} + \varepsilon_{\, sty}
}

% \newcommand{\card}{\ensuremath{\text{card\hspace{0.1em}}}}
% \DeclareMathOperator{\blackboardE}{\mathbb{E}}
% \newcommand{\expected}[1]{\blackboardE\left[#1\right]}
% \newcommand{\expectedwhen}[2]{\blackboardE_{#2}\left[#1\right]}
\newlength{\postparenlength}
\setlength{\postparenlength}{0.14em}
\newcommand{\postparen}{\hspace{\postparenlength}}

\newcommand{\varWindowDDnotes}{%
Each dot is one regression.
The $y$- and $x$-axes represent the window start and end, measured in weeks before or after the dividend day.
The dividend day starts week zero.

 The area of the circles represent the magnitude of the larger bound of the 95\% confidence interval for each regression.
In plot (a), the values are scaled by the sample mean weekly sale count.
(Using the pooled mean across Alaska and control states; see table~\ref{tab:alaska_vs_pooled_mean_sd} for details.)
Sign is the sign of the coefficient estimate. %, and because confidence intervals are symmetric, the sign of the larger bound of the 95\% confidence interval.
A standard DD regression -- where the window begins at zero and ends at the end of the period -- is the rightmost point on the $y=0$ line.
}

\begin{document}
\thispagestyle{empty}
\setcounter{page}{0}
\vspace*{0.7in plus 0.3in minus 0.3in}

\begin{center}
    {\Huge Cars in Alaska}

    \href{mailto:karldw@berkeley.edu}{\textit{\Large Karl Dunkle Werner}}$^*$

\vspace{1em}
    This draft:  \gitCommitterDate{}
  % This draft:
  %   % The \gitHash here will automatically update the URL as necessary
  %   \href{https://github.com/karldw/second_year_paper/tree/\gitHash}{
  %   % \textsc{\gitAbbrevHash{}}
  %   \gitCommitterDate{}
  %   }
    % \\
    % \href{http://karldw.org/second_year_paper.pdf}{Click here for the most recent version.}
\end{center}

\vspace{2in plus 1in minus 0.7in}

\begin{center}
    \begin{minipage}{0.7\linewidth}
        \begin{center}
            \textit{Abstract}
        \end{center}
        The Alaska Permanent Fund sends substantial dividend checks to state residents every October, possibly changing residents' decisions to purchase durable goods, including automobiles.
        Looking at a variety of outcomes from wholesale used vehicle auctions, I find small point estimates for the count, value and fuel efficiency of cars sold.
        For some specifications, I'm able to rule out changes larger than a change of a couple of percentage points.
        For others my estimates are less precise and I cannot exclude moderate-to-large effect sizes.
        These results are consistent with moderate responses to shifts in income, the permanent income hypothesis or auto dealers smoothing demand shocks with inventory.
    \end{minipage}
\end{center}

\vspace{3in plus 2in minus 1.7in}
{\footnotesize
\noindent
$^*$Many thanks to the Claire Duquennois, Elisa Duran Micco, Gabe Englander, Hal Gordon, Andy Hultgren, Stephen Jarvis, Scott Kaplan, Ben Krause, Megan Lang, Peiley Lau, Kate Pennington, Wenfeng Qiu, Eleanor Wiseman, Derek Wolfson, Jim Sallee, the faculty of the Energy Institute, Michael Anderson, Thibault Fally and Harrison Decker.
The numerous remaining errors are, of course, my own.
}
%
% \pagebreak
%
% \textsc{\Large Outline: draft for Feb 27}
%
% \begin{itemize}
%     \item Short paragraph on income effects, durable goods, why we care about people's car purchase decisions at all.
%     \item Describe APF
%     \item Contribute to some literature strands:
%     \begin{itemize}
%         \item PIH
%         \item Fuel efficiency choice
%         \item IO of used auto markets
%     \end{itemize}
%     \item Write theory of car buying, make predictions about results
%     \begin{itemize}
%         \item Explain why announcement date isn't going to be informative.
%     \end{itemize}
%
%     \item Findings -- small results
%     \begin{itemize}
%         \item Sales counts
%         \begin{itemize}
%             \item Very small coefficients, however you slice it.  Logs imply a somewhat noisier response.
%             % anticipation_window_sale_count_weekly_notitle_coef_effect_mean.pdf
%             % Then put these side-by-side:
%             % variable_window_dd_sale_count_weekly_area_conf95_max_effect_mean.pdf
%             % variable_window_dd_sale_count_log_weekly_area_conf95_max.pdf
%         \end{itemize}
%
%         \item Sales volumes
%         \begin{itemize}
%             \item  Only mention sales volumes in passing; the component pieces of sales\_count and sales\_pr are more interesting.
%             \item The effects are small when estimated in levels, moderate with logs.
%         \end{itemize}
%         \item Mean efficiency (liters per 100~km)
%         \begin{itemize}
%             % TODO in next iteration: look at mechanisms: if we control for vehicle age, what happens to the effect? What about weight?
%             \item Overall, not a large effect. We can rule out any effect larger than about 10\% of the mean, and the coefficients aren't statistically significant.
%             % anticipation_window_fuel_cons_weekly_notitle_coef_effect_mean.pdf
%             % Then put these side-by-side:
%             % variable_window_dd_sale_count_weekly_area_conf95_max_effect_mean.pdf
%             % variable_window_dd_sale_count_log_weekly_area_conf95_max.pdf
%
%         \end{itemize}
%         \item Vehicle value, as measured by one of the following:
%         \begin{itemize}
%             \item \msrp{}
%             \begin{itemize}
%                 % TODO: note that the last week of MSRP has a huge coefficient, but that says more about some year-end effect than this check.  I've dropped it to make the graph easier to see.
%                 \item  For \msrp{} the conclusions are dramatically different for levels and logs.
%                     For levels, the estimated effects are insignificant, and the confidence intervals rule out large effects -- anything larger than 10 or 20\% of the mean.
%                     % variable_window_dd_msrp_mean_weekly_area_conf95_max_effect_mean.pdf
%                     For logs, the point estimates and confidence intervals allow for much larger results.
%                     With point estimates around 0.35 and confidence interval upper bounds around 0.7 (for reasonable specifications; not those starting at the beginning), we can't reject large effects.
%                     At the same time, the lower bound of the CIs is \emph{just} above zero, so I don't want to put too much stock in the statistical significance shown in the plot.
%                     % variable_window_dd_msrp_mean_log_weekly_area_conf95_max.pdf
%             \end{itemize}
%             % \item Model-year age (TODO: add vehicle age in next draft)
%             \item Auction price (valid as a unified measure of vehicle quality if there aren't market failures)
%             \begin{itemize}
%                 \item  As with \textsc{msrp}, there's a difference between logs and levels here.  With levels, the results are insignificant, the point estimates are  close to zero and we can reject most results above 15\% of the mean price.  (The mean is \$14,912.13.)
%                 With logs, the point estimates again imply larger effects, around 0.3, with confidence intervals around 0.7.
%                 % Show two pairs here:
%                 % Pair a:
%                 % effects_individual_period_sales_pr_mean_weekly_coef_notitle.pdf
%                 % effects_individual_period_sales_pr_mean_log_weekly_coef_notitle.pdf
%                 % Pair b:
%                 % variable_window_dd_sales_pr_mean_weekly_area_conf95_max_effect_mean.pdf
%                 % variable_window_dd_sales_pr_mean_log_weekly_area_conf95_max.pdf
%             \end{itemize}
%         \end{itemize}
%         \item Show that results are robust to dropping any particular year (make a graph)
%         \item Acknowledge the issues of multiple testing -- they cut both ways, with my ``precise'' zeros being somewhat less precise.
%     \end{itemize}
%
% \end{itemize}


\pagebreak
\setcounter{page}{1}
\begin{doublespacing}
\section{Introduction}

% \RaggedRight
Automobile sales make up a large portion of the US economy, generate significant externalities and are some of the largest purchases consumers make.
For these reasons and others, we should be interested in what factors drive automobile purchases.
Since vehicles are substantial investments, it's natural to expect income effects -- when a consumer's budget constraint is relaxed (exogenously and without anticipation), their purchase decisions change.
The  Alaska Permanent Fund provides an interesting source of income variation, allowing me to look at a few aspects of the income effect in car purchases.
In the results below, I find small and statistically insignificant effects for most outcomes.

The Alaska Permanent Fund is an account set up to hold and invest a portion of the state's mineral rights revenue.
The fund has been sending dividend payments in the fourth quarter of each year to Alaskan residents since the 1980s, and in recent years the payment has happened on the first Thursday in October \parencite{hsieh2003}.
I will use this somewhat\hyp{}arbitrary payment date as a source of variation in household's income, and therefore willingness to buy vehicles.
These dividends are substantial, roughly \$1000--2000 per person, as shown in figure~\ref{fig:permanent-fund-payments-individual} \parencite{apfd_payments_summary} -- a large family may receive more than \$10,000.
(In the figure and for the rest of this article, all dollar amounts are in 2016 dollars.)
The dividend the same amount for all recipients, not contingent on age or income, and almost everyone in Alaska gets the dividend.
Major exceptions are people who have lived in the state for less than a year and people serving time for some crimes.
Approximately 91\% of the state's population applies and about 95\% of those applications are granted.

It's worth mentioning a few of the dividend details before going on.
The Permanent Fund investments are made in a broad variety of stocks and bonds, and the dividend amount is predictable.%
\footnote{The asset allocation is detailed in the Alaska Permanent Fund Corporation's website, \url{http://apfc.org/home/Content/investments/assetAllocation2009.cfm}}
The dividend is paid out of investment earnings -- 10.5\% of the past five years' earnings -- not out of present\hyp{}day oil revenue, so the amounts are correlated with financial market returns, not directly with oil prices.
The dividend amount is finalized and announced a few weeks before payment, and Alaskan newspapers publish accurate predictions before that \parencite{adn_dividend_prediction, adn_dividend_realization}.

\begin{figure}[bht]
    \caption{\large Alaska Permanent Fund payments per person}
    \label{fig:permanent-fund-payments-individual}
    \includegraphics[width=\linewidth]{permanent_fund_payments_individual_notitle.pdf}

\noindent\textsc{Notes:}
		Figures are adjusted for inflation using the annual average \textsc{cpi} \parencite{fred_inflation}.
		The 2008 total includes a one-time bonus of \$1200 (nominal dollars).
\end{figure}

The principal data source in my analysis is the Manheim wholesale used vehicle auction data.
The dataset records sale prices, vehicle identifiers, buyer and seller IDs and locations for sales between wholesalers such as auto dealerships, rental car agencies and vehicle leasing programs.
After various cleaning measures described in section~\ref{sec:manheim-data}, I have
\snippet{auctions_cleaned_total_obs_count.tex}
observations of vehicle sales,
\snippet{auctions_cleaned_alaska_obs_count.tex} in Alaska.
In the following analysis, I focus on wholesalers with billing zip codes in Alaska who buy vehicles in the auctions, and for simiplicity, I call all wholesalers ``dealers.''

The Manheim auctions occur in 32 states, but there are no auction sites in Alaska, so all of the purchases I observe for these dealers are from out of state.
While I see when the vehicles were auctioned, I don't know when they were delivered to the dealer or sold to the final customer.
Presumably, there can be some substantial delay shipping to Alaska, and I will examine delay and anticipation effects as part of my analysis.

Importantly, I don't observe final sales to customers, so I don't know the timing or their purchases or the prices they are charged, nor can I relate purchases to individuals' characteristics.
I assume that cars bought by Alaskan dealers are sold to consumers with a not\hyp{}too\hyp{}long delay (since it's costly to hold unsold cars on the lot) and not resold to another wholesaler.
The Manheim data include \emph{used} vehicles only, so I will only be able to discuss the market for new vehicles in my theoretical speculation (section~\ref{sec:model}).


% \subsection{Literature}
I mentioned that a large, \emph{unanticipated} income shock may have substantial effects.
The permanent income hypothesis literature emphasizes the importance of anticipation.
Since the Alaskan dividend timing and amount are not surprises, the strong form of the hypothesis would predict no response at all.
The permanent income hypothesis would also predict an increase in consumption when the payment is \emph{announced}.
Because the dividend date has been well known many years, I'm not able to examine the consumption decisions around a specific announcement date.
In the real world, liquidity constraints, uncertainty and behavioral factors may mean that people don't smooth their consumption perfectly.

The previous paper analyzing the Alaska Permanent Fund dividends, \textcite{hsieh2003}, used the consumer expenditure survey to look at changes in consumption from the third quarter to the fourth.
Hsieh also didn't reject the consumption smoothing hypothesis, though his identifying variation relied on potentially\hyp{}endogenous differences in family size and year\hyp{}to\hyp{}year dividend variation.
I'm unaware of other papers looking at the consumption effects of the Alaskan dividend, but many researchers have looked at tax returns.
For example, \textcite{goodman2008eitc} look at \eitc{} payments and find that \eitc{} recipients seem to spend their refunds on automobiles.

In general, I expect consumers to have heterogeneous responses to sudden shifts in their income.
Some may exhibit behavior that follows the permanent income hypothesis very closely, while others may not.
Intuitively, I expect consumers who face tighter budget constraints to have a higher propensity to shift their income -- these tight budget constraints could be because of a lack of savings or credit constraints.
Additionally, some people plan their expenditures less carefully than the permanent income hypothesis assumes; these people may feel that the dividend check is ``burning a hole in their pocket.''
Empirically, \textcite{misra2014consumption} find that, following tax rebates in 2001 and 2008, a subset of people (particularly those with high mortgage debt) increase their spending on gasoline, public transportation, health, apparel and new vehicles.




In addition to consumption smoothing, we may be interested in durable goods purchases because those goods determine energy use and generate externalities.
Some authors have argued that consumers undervalue fuel efficiency, and if so, it's possible that having more cash on hand would lead a consumer to make a longer\hyp{}sighted decision and include the full value of fuel efficiency savings.
On the other hand, more recent empirical work has found that consumers' vehicle choices include most or all of the value of saved fuel \parencite{busse2013consumers, allcott2014gasoline, grigolon2014consumer, sallee2016consumers}.



\section{Simple Model}
\label{sec:model}

\subsection{Consumers}

People value the stream of services provided by owning a vehicle, so they have preferences for vehicle traits and vehicle ownership.
Consumers are constrained by their budgets, and may therefore choose differently when they receive a check from the government.
Even though the payment is anticipated, they may not fully smooth their consumption because they don't build up enough savings in the course of their normal consumption path (if, for example, they act as buffer\hyp{}stock savers).

There are also behavioral possibilities for non\hyp{}smooth consumption.
It may be that people treating dividend checks differently than they would treat their own savings, or that people buy expensive durable goods for capricious reasons \parencite{Busse2015_weather_on_cars}.
I won't be able to disentangle these, since I don't have any situations where saved wealth is exogenously manipulated, but it's important to consider that responses may differ between this setting and other changes in consumers' wealth.

For simplicity, I'll assume each buyer purchases at most one car, indexed by $v$.
I'll also assume each vehicle's life ($L_v$) is known with certainty, and that there are no maintenance costs.
Let $\mathcal{V}_u$ and $\mathcal{V}_n$ be the set of available used and new vehicles, and $\mathcal{V}_0$ be a non\hyp{}purchase, so the consumer's full choice set is $\mathcal{V} \equiv \mathcal{V}_u \cup \mathcal{V}_n \cup \mathcal{V}_0$.
Let $g_v$ be the (endogenous) discounted gasoline expenditures associated with vehicle $v$, and $p_v$ the retail price of the vehicle.
$w$ is the wealth, including Alaskan dividends, of the consumer.
With perfect borrowing and lending markets, net wealth is $w - g_v - p_v$.
The consumer then chooses $v$ to maximize the integral of instantaneous utility:
\[
\max_{v \in \mathcal{V}} \ \ \int_0^{L_v} u(v, w - p_v - g_v) e^{-rt} dt
\]

\subsection{Auto dealers}

Economic intuition says that car dealers would not be thrilled to hold onto vehicles for longer than necessary, if only because of the opportunity cost of capital.
For instance, US News offers the advice, ``If the car has been there for more than three months, the dealer will be more anxious to sell it as quickly as possible''
\parencite{usnews_car_deals}.

To make the model less complex, I'll assume auto\hyp{}dealers don't exercise market power when they buy their used cars, that they know consumers' preferences, and that it's costly to hold inventory.
I don't exclude market power on the retail side, and since I don't have data on retail sales.
Assuming away market power is clearly a big assumption, but Alaska is a small part of the overall used car market, and making it unlikely that Alaskan buyers act as monopsonists in the auction market.

With undifferentiated products, prices $p$, quantity $q(p)$, unit cost $c$, time-to-sell $T(p)$ and interest rate $r$, the dealer's problem is:
\[
\max_p \ \ p \cdot q(p) - c \int_o^{T(p)} e^{-rt}dt
\]
With differentiated products, the problem becomes harder.
Assuming consumers buy more used cars upon receiving their dividend checks, dealers adjust.
Specifically, the dealers know what kinds of cars will be popular and will buy these beforehand, making sure they're available when the dividends arrive.
To consider the problem analytically, a dealer would consider the cross-price elasticities of demand and time-to-sell for different vehicles.
It seems likely that  the dividend checks influence both new and used car purchases, as outlined in the consumer model.
Some dealers specialize in used vehicles, while others sell both used and new vehicles.

%TODO: come up with a theoretical prediction of dealer anticipation effects


\section{Data}
\label{sec:data}

\subsection{Manheim wholesale auto auctions}
\label{sec:manheim-data}


The Manhiem auctions dataset includes a rich set of variables; I've only used the more basic ones.
The data include identifiers for buyers, sellers and auction sites, as well as the billing zip codes for all three.
I use the zip codes to identify which wholesalers are in Alaska and my control states.

\begin{table}[!hbt]
    \caption{Cleaning Manheim auction data}
    \label{tab:cleaning_manheim}
    % TODO: Make the numbers here better by having parse-numbers = true (currently broken because the snippet gets added after siunitx tries to parse the number)
    \sisetup{parse-numbers = true, round-mode=off}
\begin{tabular}{l S L{0.493\linewidth}}
    \toprule
	Elimination category & \multicolumn{1}{l}{Count removed} & Details\\
	\midrule

    Duplicate sales &
    % \snippet{clean_car_auctions_resold.tex}
    3551583
    & Same \textsc{vin} sold twice within a year.\\

    \addlinespace

    Weird vehicles &
    % \snippet{clean_car_auctions_weird_vehicles.tex}
    20257
    & Trailers, boats, air compressors, golf carts, vehicles with incomplete bodies, \textsc{atv}s and \textsc{rv}s.\\

	\addlinespace

    Bad odometer &
    % \snippet{clean_car_auctions_bad_odometer.tex}
    563074
    & Auction comments indicate odometer is flawed.\\

	\addlinespace

    Damaged &
    % \snippet{clean_car_auctions_damaged.tex}
    2520692
    & Auction comments indicate vehicle is damaged. \\

	\addlinespace

    Bad price &
    % \snippet{clean_car_auctions_price.tex}
    59251
    & Auction price seems unreasonable, outside the interval $[100, \min\{80000, 1.5\times \textsc{msrp} \}]$.\\

	\addlinespace

    Canadian &
    % \snippet{clean_car_auctions_canadian.tex}
    121965
    & Auction comments indicate vehicle is Canadian.\\

    \bottomrule
    \addlinespace
\end{tabular}
\footnotesize
\textsc{Notes:} Data are from US Manheim auctions, 2002--2014.
There are \snippet{auctions_uncleaned_total_obs_count.tex} sales before cleaning and \snippet{auctions_cleaned_total_obs_count.tex} sales after.
\snippet{auctions_cleaned_alaska_obs_count.tex} of the cleaned sales are to Alaskan buyers.
Cars that are listed for auction but not sold are not included.
Rows are removed in the order listed.
\end{table}




The auctions they include many kinds of sales outside the scope of this research.
In table~\ref{tab:cleaning_manheim}, I detail the cleaning I've done and the observation count removed in each.
To minimize the resale observations, When I observe the same vehicle (same 17-digit \vin{}) sold two or more times in a one-year period, I only use the last sale.
(This filter only addresses resales where both the initial sale and the second are in the auction data.)
Some items in the auction are a bit odd, such as trailers, boats, air compressors, golf carts, vehicles with incomplete bodies, \textsc{atv}s and \textsc{rv}s.
I exclude all of these, keeping only cars and trucks.
The Manheim data include a variable for the vehicle's odometer, as well as a field of the auction comments.
In some cases, the comments indicate that the odometer is wrong, either because it's broken or because it is a five\hyp{}digit odometer that can't record more than 100,000 miles.
I drop these as well because I'm eventually planning to include the odometer in my analysis.
Many vehicles are designated as damaged, some severely.  I drop these, since they're often not in a condition to be sold as a retail vehicle.
Some of the prices in the data seem unreasonable, either too high or too low, possibly indicating an error in data entry.
I drop cars sold for less than \$100 and cars sold for more than \$80,000 or 150\% of their $\textsc{msrp}$.
Finally, I drop cars that are flagged as Canadian, since these are presumably in a different market than US cars.
(The odometer may be measured in kilometers, there may be product differences and so on.)
Of these, only the duplicate sales and damaged vehicles eliminate a significant number of sales.


\subsection{VIN decoder}

Every modern vehicle has a unique 17-digit vehicle identification number (\vin{}).
The ID is broken in to pieces that identify the country, manufacturer, vehicle, model, trim, year and plant, plus a sequential identifier \parencite{vin_details}.
I have a database of vehicle characteristics at the country\hyp{}manufacturer\hyp{}vehicle\hyp{}model\hyp{}trim\hyp{}year level, which I match with the auction data to find the vehicles' list price (\msrp{}) and fuel economy.
% TODO: how many aren't matched?
% TODO: add N to the regression plots.

For fuel economy I use the \textsc{epa} combined rating -- a weighted average of city and highway miles per gallon -- and convert to liters per 100~km.
I do this because fuel consumption is generally much better behaved than fuel efficiency; the reciprocal can lead to unintuitive or poorly behaved estimates.
Rather than simply calculating gallons per mile, I convert to liters per 100~km.
No one uses gallons per mile, while liters per 100~km is the standard used by most other countries.

\subsection{Alaskan dividends}

For the Alaskan dividends, I've pulled the amounts from the Alaska Permanent Fund Division's website \parencite{apfd_payments_summary}.
\textcite{hsieh2003} provides the history of the dividend\hyp{}payment dates, obtained in personal correspondence with Permanent\hyp{}Fund staff.
Note that while the dividend amounts vary from year to year, the dividend is the same for all recipients in any given year.
Therefore, my year fixed effects control for the amount.


\section[DD Results]{Difference-in-Differences Results}

The auction data have before- and after\hyp{}dividend observations for treated and untreated states, so a difference\hyp{}in\hyp{}differences framework is a natural framing.
Alaska is different from many other states, so picking appropriate controls is an important and somewhat subjective endeavor.
Below, in section~\ref{sec:picking-controls} I discuss how I've chosen the current set of controls: Washington DC, Illinois, Montana, New Hampshire, New Mexico, Vermont, Washington, Wisconsin, West Virginia and Wyoming.
For the difference\hyp{}in\hyp{}differences analysis, I've chosen a default window of 70~days (10~weeks) before and after the early\hyp{}October dividend.
This fairly wide window allows for various anticipation or delayed effects without getting too close to the end of the year and associated sales slowdown.
My preferred specification aggregates the data to state\hyp{}by\hyp{}week averages or totals. Using daily data instead produces similar results that are broadly similar, but somewhat noisier.

\begin{table}[bth]
  \caption{Simple Difference-in-Differences Results (no anticipation)}
  \label{tab:standard_dd_results}
  \figureversion{tabular}
% \begin{tabular}{@{\extracolsep{0pt}}lD{.}{.}{-3} D{.}{.}{-3} D{.}{.}{-3} D{.}{.}{-3} D{.}{.}{-3} }
\begin{tabular}{@{\extracolsep{0pt}}L{8em}D{.}{.}{-3} D{.}{.}{-3} D{.}{.}{-3} D{.}{.}{-3} D{.}{.}{-3} }
% \\[-1.8ex] & \multicolumn{5}{c}{post + alaskan\_buyer\_post + pfd\_payment\_X\_alaska} \\
% & \multicolumn{1}{c}{(1)} & \multicolumn{1}{c}{(2)} & \multicolumn{1}{c}{(3)} & \multicolumn{1}{c}{(4)} & \multicolumn{1}{c}{(5)}\\
& \multicolumn{1}{c}{Sale total} & \multicolumn{1}{c}{Sale count} & \multicolumn{1}{c}{Log sale count} & \multicolumn{1}{c}{Sales price} & \multicolumn{1}{c}{MSRP}\\

\midrule \\
 % Post & -872,501.000^{***} & -67.900^{***} & -0.257^{***} & -91.113 & -231.032 \\
  % & (188,441.300) & (13.870) & (0.042) & (207.647) & (303.504) \\
Alaskan buyer $\times$  & 680.^{***} & 56.8^{***} & -0.121 & -103. & 648. \\
post  & (190.) & (14.1) & (0.129) & (550.) & (744.) \\
  \addlinespace
 \textsc{apfd} payment $\times$  & -0.123 & -0.021 & 0.0003^{***} & 0.847 & 2.02^{***} \\
Alaska  & (0.331) & (0.022) & (0.0001) & (0.487) & (0.773) \\
  \addlinespace
\midrule
Observations & \multicolumn{1}{c}{$1092$} & \multicolumn{1}{c}{$1092$} & \multicolumn{1}{c}{$1092$} & \multicolumn{1}{c}{$1092$} & \multicolumn{1}{c}{$1092$} \\
% R$^{2}$ & \multicolumn{1}{c}{0.808} & \multicolumn{1}{c}{0.837} & \multicolumn{1}{c}{0.846} & \multicolumn{1}{c}{0.270} & \multicolumn{1}{c}{0.373} \\
Adjusted R$^{2}$ & \multicolumn{1}{c}{$0.805$} & \multicolumn{1}{c}{$0.835$} & \multicolumn{1}{c}{$0.844$} & \multicolumn{1}{c}{$0.258$} & \multicolumn{1}{c}{$0.363$}\\ \bottomrule
\addlinespace
\end{tabular}

{\footnotesize
\textsc{Note:}
Observations are state-by-week totals and averages.
All regressions include state and year fixed effects. Standard errors are clustered at the intersection of state-by-year.
DD Control states are the 10 selected in section~\ref{sec:picking-controls}.
Sale total is the total value of vehicles sold, in thousands of dollars.
The average value of the \textsc{apfd} payment is \$1656.
$^{**}p<0.05$; $^{***}p < 0.01$.
}
\end{table}


\subsection{Picking control states}
\label{sec:picking-controls}
In considering which states are good controls for Alaska, it's worth diving a little more into the structure of the Manheim auction data.
Auctions occur in 32 states and Puerto Rico, and Alaska is not among them.
Therefore all purchases by Alaskan buyers are at out\hyp{}of\hyp{}state auctions.


For the difference\hyp{}in\hyp{}differences below, I've chosen Washington DC, Illinois, Montana, New Hampshire, New Mexico, Vermont, Washington, Wisconsin, West Virginia and Wyoming Idaho, Oregon and Utah.
I picked these states by first estimating the \textcite{DoudchenkoImbens2016DD} for several outcome variables, then comparing the weights assigned to various states.%
\footnote{A more formal approach would minimize prediction error across all outcomes simultaneously.}
Ultimately, the uniform weights on the 10 states I've chosen seem to predict similarly to the synthetic controls weights -- see figure~\ref{fig:synth-weights}

The choice is somewhat subjective -- I chose these states because they have a large fraction of their out\hyp{}of\hyp{}state purchases in the auction states where Alaskan dealers are buying.
This heuristic is appealing because it captures the markets where Alaska is operating, as well as additional factors, such as transportation costs, that arise from dealers buying their vehicles out of state.
% See the code in find_match_states_crude_unmemoized for full details..
A more robust approach is to use best\hyp{}subset or synthetic controls methods described in \textcite{DoudchenkoImbens2016DD}; I plan to implement these in a future iteration of this project.

Beyond arbitrarily picking control states, one can use synthetic controls methods to pick an appropriate control.
The original synthetic controls approach, proposed by \textcite{abadie2010synthetic} , is compelling, but in some ways it's limiting.
In \textcite{DoudchenkoImbens2016DD}, the authors bring a broader framework that nests the Abadie-Diamond-Hainmueller synthetic controls and standard difference-in-differences.
In particular, they highlight two additional approaches, which they call \textit{constrained regression} and \textit{best subset selection.}
I've chosen to implement the constrained regression (equation 5.6 in Doudchenko and Imbens' paper), which is much like difference-in-differences, but rather than giving all control states equal weight, the algorithm picks state-specific weights in the $[0, 1]$ interval.
I've constrained the weights to sum to $1$ and allowed for a non-zero difference in means, estimating weights that minimize the sum of squared residuals in the 10-week period before the dividend days. (All years are pooled.)

Allowing for a difference in means is appealing, since Alaska is a small market and I don't want to spend all of the model's degrees of freedom to match means.
As with difference-in-differences, I'm interested in changes of trend.
Last, I've added a small $l_0$ penalty, minimizing the number of non-zero weights.
While the results may depend on these choices of constraints and meta-parameters, time constraints prevent me from fully exploring the system's sensitivity.

\begin{table}[bth]
\caption{Weights on Control States}
\label{tab:synth-weights}
\centering
\begin{minipage}{0.763\linewidth}

\figureversion{tabular}
\begin{tabular}{l S S S S S}
State & \multicolumn{1}{l}{Sales count} & \multicolumn{1}{l}{Fuel cons.} &
 \multicolumn{1}{l}{Sales price} &
  \multicolumn{1}{l}{MSRP} & \multicolumn{1}{l}{DD chosen} \\
\midrule
% Table
\textsc{az} & 0.0 & 0.0011940 & 0.0 & 0.0 & 0.0 \\
\textsc{ca} & 0.0 & 0.0 & 0.0880577 & 0.0 & 0.0 \\
\textsc{dc} & 0.0 & 0.0634228 & 0.0279254 & 0.0 & 0.1 \\
\textsc{de} & 0.0529473 & 0.0 & 0.0 & 0.0 & 0.0 \\
\textsc{hi} & 0.0274633 & 0.0 & 0.0 & 0.0 & 0.0 \\
\textsc{il} & 0.0 & 0.0 & 0.2460708 & 0.1360270 & 0.1 \\
\textsc{in} & 0.0199278 & 0.0 & 0.0 & 0.0 & 0.0 \\
\textsc{me} & 0.0356947 & 0.0 & 0.0 & 0.0 & 0.0 \\
\textsc{mt} & 0.0 & 0.0 & 0.2671123 & 0.1604908 & 0.1 \\
\textsc{nh} & 0.0 & 0.1906009 & 0.0 & 0.1504336 & 0.1 \\
\textsc{nm} & 0.0 & 0.0 & 0.3708337 & 0.0 & 0.1 \\
\textsc{vt} & 0.6939342 & 0.0 & 0.0 & 0.0 & 0.1 \\
\textsc{wa} & 0.0 & 0.1927583 & 0.0 & 0.3172060 & 0.1 \\
\textsc{wi} & 0.0 & 0.0 & 0.0 & 0.2358426 & 0.1 \\
\textsc{wv} & 0.1082959 & 0.1710181 & 0.0 & 0.0 & 0.1 \\
\textsc{wy} & 0.0617368 & 0.3810059 & 0.0 & 0.0 & 0.1 \\
\bottomrule
\addlinespace
\end{tabular}

{\footnotesize
\textsc{Note:}
Only states that receive positive weight in any of the constrained regressions are shown.
Each column represents one constrained regression, and the values are the weight assigned to a particular state.
In the DD chosen column, I've indicated the 10 states that will be used as (equally-weighted) controls in the remainder of the analysis.
These are the 10 states that received the greatest sum of weights across the four constrained regressions shown.
}
\end{minipage}
\end{table}

In addition, time constraints lead me to depart from \textcite{DoudchenkoImbens2016DD}.
Instead of following their advice on statistical inference, I am choosing to take the set of states chosen by synthetic controls and use them (with equal weights) in a difference-in-differences estimation.%
\footnote{In retrospect, a best-subset approach may have been better suited, but as I mentioned, time is fleeting.}
Doing so allows me to more directly estimate standard errors, to use the same weights across different outcomes and to more easily include regressors.
Table~\ref{tab:synth-weights} and figure~\ref{fig:synth-weights} provide comparisons of the control-state weights.


\begin{figure}
    \caption{Synthetic Control Weights Demonstrate Similar Trends}

    \label{fig:synth-weights}
    \includegraphics[width=\textwidth]{synth_cross_var_weights_comparison.pdf}
    {\footnotesize
    \textsc{Note:}
    The two plots show standardized outcomes over event time, taking averages across all years.
    The lines in each plot show the predicted value of the outcome (sale count or fuel consumption) where the control state weights were chosen looking looking at a particular variable (sales count, fuel consumption or sale price).

    The the solid black line is the truth.
    The DD weight (dashed orange line) assigns equal weight to 10 states, as described in table~\ref{tab:synth-weights}.
    The rest of the lines represent predicted values from minimizing the constrained regression the indicated variable.
    We might expect the sales count line to fit the sales count outcome particularly well in weeks before the event date.
    Likewise, we might expect the fuel consumption line to fit the fuel consumption outcome particularly well.
    Neither of these seem to be the case.
    In fact, the DD weight line seems to follow about as well as the rest.

    }
\end{figure}

% \begin{figure}[bth]
% 	\caption{Alaska-only and pooled mean and standard deviations}
% 	\label{fig:alaska_vs_pooled_mean_sd}
% 	\includegraphics[width=\linewidth]{comparison_alaska_vs_pooled_mean_sd.pdf}
% \end{figure}


\begin{table}[bth]
	\caption{Alaska-only and Pooled Mean and Standard Deviations}
	\label{tab:alaska_vs_pooled_mean_sd}
    \figureversion{tabular}
    \begin{tabular}{L{7.3em} S S S S S}
        Variable & \multicolumn{1}{l}{\textsc{ak} mean} & \multicolumn{1}{l}{\textsc{ak} \textsc{sd}} & \multicolumn{1}{l}{Control mean} & \multicolumn{1}{l}{Control \textsc{sd}} & \multicolumn{1}{l}{Diff.\ means} \\
        \midrule
        Cars sold\par (weekly)                 & 423.301886792453           & 148.088926624104           & 7761.46140651801           & 2575.18254019928           & -7338.15951972556         \\
        \addlinespace
        Fuel consumption\par (L/100km) & 13.5318989830821           & 0.297873941875563          & 12.5950143489404           & 0.171468103633867          & 0.9368846341417           \\
        \addlinespace
        Sales price        & 16385.7464853612           & 1605.30178194432           & 13213.6287600248           & 1466.32534834483           & 3172.1177253364           \\
        \addlinespace
        MSRP                       & 29315.4287041851           & 1082.29042624375           & 26654.6421263061           & 646.750532645365           & 2660.786577879            \\
        \addlinespace
        Sale total \par (\$ millions)                & 7.00527505401487           & 2.74497385552374           & 103.625361134693           & 36.2120750339191           & -96.6200860806781         \\
        \addlinespace
        Population\par (thousands)     & 692.414769230769           &                            & 3278.19153076923           &                            & -2585.77676153846         \\
        \addlinespace
        GDP per capita             & 75742.8503015557           &                            & 65700.4372354834           &                            & 10042.4130660723          \\
        %    Cars sold                  & 423.301886792453           & 148.088926624104           & 7761.46140651801           & 2575.18254019928           & -7338.15951972556         \\
   Fuel consumption (L/100km) & 13.5318989830821           & 0.297873941875563          & 12.5950143489404           & 0.171468103633867          & 0.9368846341417           \\
   Average sales price        & 16385.7464853612           & 1605.30178194432           & 13213.6287600248           & 1466.32534834483           & 3172.1177253364           \\
   MSRP                       & 29315.4287041851           & 1082.29042624375           & 26654.6421263061           & 646.750532645365           & 2660.786577879            \\
   Sale total                 & 7.00527505401487           & 2.74497385552374           & 103.625361134693           & 36.2120750339191           & -96.6200860806781         \\
   Population (thousands)     & 692.414769230769           &                            & 3278.19153076923           &                            & -2585.77676153846         \\
   GDP per capita             & 75742.8503015557           &                            & 65700.4372354834           &                            & 10042.4130660723          \\

        \bottomrule
        \addlinespace
    \end{tabular}
    \figureversion{proportional}

    {\footnotesize
    \textsc{Note:}
    Cars sold is a weekly average; Sale total is millions of dollars per week and Population is thousands of people. All dollar figures are measured in 2016 dollars.
    Control states are Washington DC, Illinois, Montana, New Hampshire, New Mexico, Vermont, Washington, Wisconsin, West Virginia and Wyoming -- with equal weights -- as detailed in table~\ref{tab:synth-weights}.
    }

\end{table}


In addition to considering the control group for difference\hyp{}in\hyp{}differences, one might also worry about the effect size measurements.
In some of the analyses below, I divide by the sample mean or standard deviation to look at the size of the estimated coefficient.
Table~\ref{tab:alaska_vs_pooled_mean_sd} shows the mean and standard deviation I use for these normalizations.
Using the event window sample (70~days before to 70~days after the dividend) is almost identical to the full\hyp{}year sample, but Alaska is a smaller market than the controls I've chosen.
Alaska has a lower mean and standard deviation in cars sold, and therefore the total value of sales.
As well, the cars sold in Alaska are a bit more expensive and a bit less efficient.
It's worth keeping these differences in mind in the outcomes below.
The log specifications adjust for percentage differences more directly; I don't need to divide by the mean.



\subsection{Picking an anticipation window}

As mentioned in section~\ref{sec:model}, we might expect consumers to sharply change their consumption when they receive the dividend checks.
Auto dealers would anticipate that behavior and change their auction purchases somewhat earlier, depending on transportation time and the costs of holding inventory.

Theory doesn't tell us, however, how long that anticipation window should be.
To be flexible, I have estimated the results for varying windows from one to nine weeks of anticipation, as shown in figure~\ref{fig:anticipation_window_sale_count}.
In all cases, the anticipation window ends the immediately before the dividend day.

On the other hand, dealers anecdotally say they just try to maintain their inventories.
While the dealers know that customers tend to  buy more cars around the dividend date and when they receive tax refunds, but the dealers report that don't make additional efforts to stock up before the dividend is sent.
In this case, the auction data should show no anticipation effect.
Instead, there should be some increase in auction sales after the dividend, as customers buy cars and deplete dealers' inventories, which the dealers then restock.
On the other hand, one might put more credence in the anecdotes.
To show that there is still no effect on vehicle sales, I'll estimate an even more flexible treatment window.
In this style of plot, I estimate all possible windows longer than one week and plot out the confidence interval for each, as in figure~\ref{fig:variable_window_sale_count}.




% \begin{figure}[!hbt]
%     \caption{Treatment coefficients on Alaska $\times$ anticipation: total sales}
%     \label{fig:anticipation_window_sale_tot}
%     \includegraphics[width=\linewidth]{anticipation_window_sale_tot_weekly_notitle_pooled_sd.pdf}
%
%     {\footnotesize
%     The plot shows the Alaska $\times$ anticipation coefficients ($\gamma_2$) from the regression in equation~\ref{eqn:cars_sold_with_anticipation}, with varying anticipation windows.
%     Error bars indicate 95\% confidence intervals.
%     The horizontal line at \$\snippet{sales_tot_weekly_thousands_std_dev.tex} is the standard deviation of weekly sales volume (in thousands).
%     All windows end at $-1$ week.
%     }
% \end{figure}


\subsection{Difference-in-differences equations}
Let $t$ represent time, measured in event weeks, $t \in [-10, 10]$.
The week the dividend is received is $t=0$.
Let $s$ index states and $y$ index years: $s \in \{\textsc{ak}, \textsc{id},\textsc{mt},\textsc{ut} \}$ and $y \in [2002, 2014]$.
Let $W_0$ be the start of the anticipation window and $W_1$ be the end, so $[W_0, W_1]$ is the window interval and $\anticipation$ is a dummy for $t$ in the window.
Though I'm calling it an ``anticipation window,'' I'm going to estimate some versions with $W_0$ or $W_1 > 0$.
For brevity, let $F_{sty}$ be the vector of state and year fixed effects, four for states and 13~for years.
The capital Greek letter premultiplying $F$ represents a vector of the 17~coefficients.
In a slight abuse of notation, let $\varepsilon$ be the error term in each of the equations below, though it represents a different error term in each.
The difference\hyp{}in\hyp{}differences specification is the same in all eight equations, but the left\hyp{}hand side is different in each.
% Listing out all of the outcomes in gory detail, we have the following.
The terms to be estimated are, for $i = 1,2,3,4$,
$\alpha_i$ and $A$ (eqn.~\ref{eqn:cars-sold});
$\beta_i$ and $B$ (eqn.~\ref{eqn:log-cars-sold});
$\gamma_i$ and $\Gamma$ (eqn.~\ref{eqn:sale-price});
% $\delta_i$ and $\Delta$ (eqn.~\ref{eqn:log-sale-price});
$\zeta_i$ and $Z$ (eqn.~\ref{eqn:msrp}); and
% $\eta_i$ and $H$ (eqn.~\ref{eqn:log-msrp});
$\theta_i$ and $\Theta$ (eqn.~\ref{eqn:fuel-cons}).
% $\lambda_i$ and $\Lambda$ (eqn.~\ref{eqn:log-fuel-cons}).
To provide one example, the equation for cars sold is:
\begin{align}
    \label{eqn:cars-sold}
    \ddEqn{Cars sold}{\alpha}{A}
\end{align}
%     %
%     \label{eqn:log-cars-sold}
%     \ddEqnLog{Cars sold}{\beta}{B} \\
%     %
%     \label{eqn:sale-price}
%     \ddEqn{Sale price}{\gamma}{\Gamma}\\
%     %
%     \label{eqn:log-sale-price}
%     \ddEqnLog{Sale price}{\delta}{\Delta} \\
%     %
%     \label{eqn:msrp}
%     \ddEqn{\msrp{}\!}{\zeta}{Z} \\
%     %
%     \label{eqn:log-msrp}
%     \ddEqnLog{\msrp{}\!}{\eta}{H} \\
%     %
%     \label{eqn:fuel-cons}
%     \ddEqn{Fuel cons}{\theta}{\Theta} \\
%     %
%     \label{eqn:log-fuel-cons}
%     \ddEqnLog{Fuel cons}{\lambda}{\Lambda}
%     % Greek letters that could be useful here, in order:
%     % \alpha, \beta, \gamma, \delta, \zeta, \eta, \theta, \iota, \kappa, \lambda, \mu,\nu, \xi, \phi, \chi, \psi, \omega
% \end{align}

\subsection{Volume of cars sold}


\begin{figure}
    \caption{Quarterly Manheim Sales vs.\ Total Registrations}
    \label{fig:comparison_regs_auctions_sales}
    \includegraphics[width=\linewidth]{combined_regs_sales_counts_per_capita_alaska_vs_no_resale_quarterly_notitle.pdf}

    {\footnotesize
    \textsc{Source:} Registrations are from R.L.Polk \& Co., covering new and used car and light truck registrations from 2000 to 2011.
    Auctions are from the Manheim used-vehicle auction data, covering 2002--2014.
    The vehicle types included are detailed in table~\ref{tab:cleaning_manheim}.

    }
\end{figure}
\begin{figure}[!hbt]
    \caption{$\alpha_2$ and $\beta_2$, the treatment coefficients on Alaska $\times$ anticipation, for varying windows}
    \label{fig:anticipation_window_sale_count}
    \begin{subfigure}{\linewidth}
    \includegraphics[width=\linewidth]{anticipation_window_sale_count_weekly_notitle_coef_effect_mean.pdf}
    \end{subfigure}
    \begin{subfigure}{\linewidth}
    \includegraphics[width=\linewidth]{anticipation_window_sale_count_log_weekly_notitle_coef.pdf}
    \end{subfigure}

    {\footnotesize
    \textsc{Notes:}
    The $x$-axis represents event weeks, estimating the difference\hyp{}in\hyp{}differences coefficients for different window starts (varying $W_0$).
    The coefficients on top are from equation~\ref{eqn:cars-sold} and the bottom are from equation~\ref{eqn:log-cars-sold}.
    All windows end at $W_1 = -1$ event weeks.
    The normalization for the top chart, mean weekly sale count, is \snippet{sales_count_weekly_mean.tex} cars per state per week.
    }
\end{figure}


Let's first consider the most direct outcome variable, the number of cars sold.
Note that the effect on used car sales is theoretically ambiguous.
If the permanent income hypothesis holds, there should be no change.
If not, I expect overall car sales to increase, but it could be that people purchase more new cars and fewer used cars.
\begin{align}
    \tag{\ref{eqn:cars-sold}}
    \ddEqn{Cars sold}{\alpha}{A}\\
    %
    \label{eqn:log-cars-sold}
    \ddEqnLog{Cars sold}{\beta}{B}
\end{align}

Figure~\ref{fig:anticipation_window_sale_count} shows these first results, in levels and logs.
None of the estimates of $\alpha_2$ or $\beta_2$ are statistically significant.
The levels estimates ($\alpha_2$) are fairly precise, allowing me to reject an effect bigger than 2\% of the mean.
(Effect sizes measured in standard deviations, rather than as a fraction of the mean, are also small.)
Alaska and the control states have somewhat different means, as shown in the comparison of table~\ref{tab:alaska_vs_pooled_mean_sd}, but the sale count effect is small relative to both the pooled and Alaska\hyp{}specific mean.
I default to calculating the pooled mean across Alaska and the control states and across all weeks in the data.
Calculating the mean on only periods within the event window -- from 70~days before the dividend is sent to 70~days after -- gives similar results as the all\hyp{}year sample.
The log regression, estimates of $\beta_2$, are also not huge, but the standard errors are wide enough that I can't rule out larger effects.

\begin{figure}[!hbt]
    \caption{Sales count effects with varying windows}
    \label{fig:variable_window_sale_count}
    \begin{subfigure}{0.97\linewidth}
        \caption{Estimate in levels ($\alpha_2$), scaled by weekly mean sales}
        \centering
        \includegraphics[width=\linewidth]{variable_window_dd_sale_count_weekly_area_conf95_max_effect_mean.pdf}
    \end{subfigure}
    \begin{subfigure}{0.97\linewidth}
        \caption{Estimate in logs ($\beta_2$)}
        \includegraphics[width=\linewidth]{variable_window_dd_sale_count_log_weekly_area_conf95_max.pdf}
    \end{subfigure}

    {\footnotesize
    \textsc{Notes:}
    \varWindowDDnotes
    }
\end{figure}

As mentioned above, it's possible that dealers aren't anticipating sales, instead reacting to customer demand and only buying new cars as necessary.
This behavior wouldn't lead to any anticipation effect, but might lead to increased sales after the dividend date.

To show that there is still no effect on vehicle sales, I'll estimate a more flexible treatment window.
In figure~\ref{fig:variable_window_sale_count}, I've estimated equations~\ref{eqn:cars-sold} and~\ref{eqn:log-cars-sold} for all possible windows longer than one week.
(I've excluded windows that span more than 19 of the 21~weeks in the event period, as these are both irrelevant and imprecisely estimated.)
The area of each point is scaled to the 95\% confidence interval, since this is the largest value we can tell an economic story about.
I only plot one bound of the 95\% confidence interval in each point -- I've chosen the bound with the larger magnitude.
For negative point estimates I've plotted the lower bound of the confidence interval and for positive point estimates I've plotted the upper bound.
In the graph, points that are darker (less transparent) represent point estimates that are statistically significant.
There is no multiple\hyp{}testing adjustment to the confidence intervals in the graph.
Adjusting for the number of tests would increase the intervals, leading my small zeros to be somewhat less small.

As in the anticipation window plot, most of the estimates are small and statistically insignificant.
The levels estimates are tiny across the board, allowing me to reject most effects larger than 2\% of the mean car sales.
The log estimates are noisier, but for most windows I can reject an effect larger than 30\% or so.
It's curious that for windows beginning and ending late in the event period, the levels estimates are positive while the log estimates are negative.

In addition to the number of cars sold, we could look at the dollar value of total sales.
I've run those regressions, but I find the component pieces -- the number of cars and the price of each -- more interesting, so I won't linger on the total sales volume analysis.
As with sales count, the effects are quite small when estimated in levels, with wider confidence intervals when estimated in logs.

\subsection{Quality of cars sold: sale price and MSRP}
Though there weren't huge effects in the number of cars sold, one might also anticipate a change in vehicle prices or quality.
With the very\hyp{}detailed auction data, there are many possible measures of quality: age, odometer reading, luxury car status and so on.
For this analysis, I'll focus on the manufacturer's suggested retail price (\msrp{}) and sales price.
The \msrp{} is fixed for each car, so it captures a measure of the original luxuriousness.
The sales price may change over time, because of vehicle depreciation or market dynamics.
Alaskan buyers are a small segment of the market, so I'm assuming their purchases don't move around prices.
If that's the case, and if there are no other market failures, sales price measures the depreciated vehicle quality.
(In practice, auction participants have some behavioral biases, including discontinuous valuation around round odometer readings; \cite{sallee2016consumers}.)
For brevity, I will omit the anticipation window plots (in the style of figure~\ref{fig:anticipation_window_sale_count}) and only present the dot plots.


The equations for price and \msrp{} are:
\begin{align}
    \label{eqn:sale-price}
    \ddEqn{Sale price}{\gamma}{\Gamma}\\
    %
    % \label{eqn:log-sale-price}
    % \ddEqnLog{Sale price}{\delta}{\Delta}\\
    %
    \label{eqn:msrp}
    \ddEqn{\msrp{}\!}{\zeta}{Z} %\\
    %
    % \label{eqn:log-msrp}
    % \ddEqnLog{\msrp{}\!}{\eta}{H}
\end{align}


Sales price effects are noisier, relative to their mean, than the sales count effects.
The point estimates are still small, around 5\% of the mean, but the confidence intervals don't rule out effects as large as 20 or 30\%.
In logs, the point estimates are larger, indicating a 30\% increase, and the confidence intervals don't rule out effects as large as 100\% price increases.

For \msrp{} the conclusions are dramatically different for levels and logs.
    For levels, the estimated effects are insignificant, and the confidence intervals rule out large effects -- anything larger than 10 or 20\% of the mean.
    For logs, the point estimates and confidence intervals allow for much larger results.
    With point estimates around 0.35 and confidence interval upper bounds around 0.7 (for reasonable specifications; not those starting at the beginning), we can't reject large effects.
    At the same time, the lower bound of the confidence intervals is \emph{just} above zero, so I don't want to put too much stock in the statistical significance shown in the plot.


\subsection{Efficiency of cars sold}

Why look at efficiency?
% TODO: talk about the purchasing behavior of people in Alaska: trucks etc
Some people have speculated that there's an energy efficiency gap, where people are failing to appreciate the energy efficiency implications of their purchases.
If that's true, it may be that the dividend check provides liquidity and allows the consumer to more optimally trade off current and future costs.
More recent literature has found that people buy incorporate much \parencite{allcott2014gasoline} or all \parencite{busse2013consumers, grigolon2014consumer, sallee2016consumers} of the expected discounted fuel expenditures in their vehicle purchase decisions.

As \textcite{kiso2013automobilefueleconomy} reminds us, fuel economy is intimately related to other vehicle characteristics, so even if consumers are already pricing in fuel costs, having more money to spend on a vehicle may change the fuel economy.
Since burning fuel generates externalities, the question still matters. %TODO: flesh this out?
\begin{align}
    \label{eqn:fuel-cons}
    \ddEqn{Fuel cons}{\theta}{\Theta}% \\
    %
    % \label{eqn:log-fuel-cons}
    % \ddEqnLog{Fuel cons}{\lambda}{\Lambda}
\end{align}
In the difference\hyp{}in\hyp{}differences analysis, the effect is rather small.
I can rule out any effect larger than about 10\% of the mean, and the coefficients aren't statistically significant.
The effect in logs is slightly larger, with point estimates around 10\% and confidence intervals around 30\%.



\section{Planned Extensions}
%
% \subsection{Synthetic controls and standard errors}
% \label{sec:synthetic-controls}
% As I discussed in section~\ref{sec:picking-controls},  I picked control states that seemed to be appropriate controls for Alaska.
% These control states participate in similar markets and, as Alaska does, they buy many of their vehicles in out\hyp{}of\hyp{}state auctions.
% In future iterations of this research project, I plan to use a synthetic control.
% I'll also consider the more general frameworks presented by
% \textcite{DoudchenkoImbens2016DD} and \textcite{Xu2016}.
%
% The standard errors I've used and discussed are only robust to heteroskedasticity, not any form of autocorrelation.
% Therefore, the \textcite{bertrand_duflo2004DD} critique applies, and my inferences of narrow confidence intervals are somewhat overstated.
% Clustering at the state level would be inappropriate because I'm only using a few states, but I can also draw inferences from the synthetic controls without clustering on individual states.
%
% \subsection{Robustness}
% Not all years are the same.
% A natural robustness check is to exclude one year at a time and see if the results change dramatically.

% \subsection{Other measures of vehicle quality}
% \begin{itemize}
%     \item Age
%     \item Depreciated value
% \end{itemize}

\subsection{Consumer side}
Many of my analyses and inferences are limited by using wholesale data.
Without more information on consumers, it's hard to draw conclusions about timing or welfare.
I've obtained a quarter\hyp{}by\hyp{}zip code database of registration counts, but much more detailed datasets are available.
%
\subsection{Auto market IO}
One of the strongest aspects of this dataset is to examine the behavior of buyers and sellers in the market, rather than focusing on sales to end-consumers.
I plan to look further into the specifics of buyers and sellers, looking at their behavior over time instead of just their state.
It's possible that buyer--seller interactions are important, or that certain buyers and sellers respond differently to the dividend.
Finally, I plan to look in more detail at the price effects I've assumed away, examining if dealers in Alaska pay more for the same type of vehicle in years with larger dividends.
%
% \begin{itemize}
%     \item Buyer-dealer interactions and market power
%     \item Repeating the while analysis at the buyer\hyp{}week or buyer category\hyp{}week level.
%     \item Price effects -- dealers in Alaska paying more for the same \vin10 in years with high dividends.
%     % To get that right, it's important to control for vehicle quality carefully; a higher price could indicate market shifts, or just that Alaskan dealers are shifting to newer used cars.
%     % (Both are interesting, but they're different stories.)
%
% \end{itemize}


\section{Conclusion}

In the analyses above, I've found mostly\hyp{}small and generally insignificant effects.
Looking at changes in sales counts, prices, \msrp{} and fuel consumption, I can rule out large effects.
On one hand,  the results are unsurprising.
The dividend is known in advance by both dealers and consumers.
Buyers who are not liquidity constrained may not change their behavior at all, and dealers may adjust their inventory smoothly.
On the other, consumers have been shown to be mercurial \parencite{Busse2015_weather_on_cars}.
In future research, there are a wide variety of extensions and further investigation using this dataset.

%
% \begin{itemize}
%     \item Generally, not huge effects.
%     \item Mention the outcomes specifically.
%     \item Why would that be?
%     \begin{itemize}
%         \item Possible that people aren't responding in a dramatic way
%         \item Possible that consumers \emph{are} responding dramatically, but the shifts are absorbed in inventory, which the dealers gradually replenish.
%
%     \end{itemize}
% \end{itemize}

\FloatBarrier


\pagebreak
% Make a much more compressed format for the references: single spaced, 1in margin.
% fancyhfoffset makes fancyhdr recalculate the footer location for the new margin.
\end{doublespacing}
\newgeometry{margin=1in}
\fancyhfoffset[O]{0pt}


% References
\printbibliography

\begin{refsection}[software.bib]
\nocite{*}
\printbibliography[heading=subbibliography, title={Software Used}]
\end{refsection}


\pagebreak
\appendix

\section{Event-Window Figures}


\begin{figure}[!hbt]
    \caption{Sales price effects with varying windows}
    \label{fig:variable_window_sales_price}
    \begin{subfigure}{0.97\linewidth}
        \caption{Estimate in levels ($\gamma_2$), scaled by weekly mean sales price}
        \includegraphics[width=\linewidth]{variable_window_dd_sales_pr_mean_weekly_area_conf95_max_effect_mean.pdf}
    \end{subfigure}
    % \begin{subfigure}{0.97\linewidth}
    %     \caption{Estimate in logs ($\delta_2$)}
    %     \includegraphics[width=\linewidth]{variable_window_dd_sales_pr_mean_log_weekly_area_conf95_max.pdf}
    % \end{subfigure}

    {\footnotesize
    \textsc{Notes:}
    \varWindowDDnotes
    }
\end{figure}

\begin{figure}[!hbt]
    \caption{MSRP effects with varying windows}
    \label{fig:variable_window_msrp}
    \begin{subfigure}{0.97\linewidth}
        \caption{Estimate in levels ($\zeta_2$), scaled by weekly mean MSRP}
        \includegraphics[width=\linewidth]{variable_window_dd_msrp_mean_weekly_area_conf95_max_effect_mean.pdf}
    \end{subfigure}
    % \begin{subfigure}{0.97\linewidth}
    %     \caption{Estimate in logs ($\eta_2$)}
    %     \includegraphics[width=\linewidth]{variable_window_dd_msrp_mean_log_weekly_area_conf95_max.pdf}
    % \end{subfigure}

    {\footnotesize
    \textsc{Notes:}
    \varWindowDDnotes
    }
\end{figure}

\begin{figure}[hbtp]
    \caption{Fuel consumption effects with varying windows}
    \label{fig:variable_window_fuel_cons}
    \begin{subfigure}{0.97\linewidth}
        \caption{Estimate in levels, scaled by weekly mean sales}
        \includegraphics[width=\linewidth]{variable_window_dd_fuel_cons_weekly_area_conf95_max_effect_mean.pdf}
    \end{subfigure}
    % \begin{subfigure}{0.97\linewidth}
    %     \caption{Estimate in logs}
    %     \includegraphics[width=\linewidth]{variable_window_dd_fuel_cons_log_weekly_area_conf95_max.pdf}
    % \end{subfigure}

    {\footnotesize
    \textsc{Notes:}
    \varWindowDDnotes
    }
\end{figure}

\end{document}
