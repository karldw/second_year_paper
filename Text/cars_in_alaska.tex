\documentclass[11pt,letterpaper,oneside]{article}
\usepackage[utf8]{inputenc}
\usepackage[T1]{fontenc}
\usepackage{amssymb,amsmath,amsthm}
\usepackage{enumitem}
\usepackage[mathlf]{MinionPro}
\usepackage{carlito}
% \usepackage{roboto}
\usepackage{inconsolata}
\usepackage{sectsty}
\usepackage{setspace}
\usepackage[font=sf]{caption}
\allsectionsfont{\sffamily\mdseries}
\usepackage[activate={true,nocompatibility},final,tracking=true,spacing=true]{microtype}
\microtypecontext{spacing=nonfrench}
\usepackage{typearea}   % Let the typearea package work out the best margins (wider than the defaults)
\usepackage{fancyhdr}
\pagestyle{fancy}
\lhead{}
\chead{}
\rhead{}
\lfoot{}
\cfoot{}
\rfoot{\thepage}
\renewcommand{\headrulewidth}{0pt}
\renewcommand{\headheight}{10pt}

\newcommand{\proofsep}{\vspace{-0.75em}}
\usepackage{booktabs}

\usepackage{graphicx}
\usepackage{subcaption}
\usepackage{placeins}  % use \usepackage[section]{placeins} to automatically put a FloatBarrier before each section
% Improve math spacing around |,\left and \right
% See: https://tex.stackexchange.com/questions/2607/spacing-around-left-and-right/
\let\originalleft\left
\let\originalright\right
\renewcommand{\left}{\mathopen{}\mathclose\bgroup\originalleft}
\renewcommand{\right}{\aftergroup\egroup\originalright}

% Remove section numbering
\makeatletter
% we use \prefix@<level> only if it is defined
\renewcommand{\@seccntformat}[1]{%
    \ifcsname prefix@#1\endcsname
    \csname prefix@#1\endcsname
    \else
    \csname the#1\endcsname\quad
    \fi}
% define \prefix@section

%\newcommand\prefix@section{}
%\newcommand\prefix@subsection{}
%\newcommand\prefix@subsubsection{}
\makeatother

% Define new commands for "such that", the blackboard-bold letters and "closure".
\newcommand{\st}{\textit{ s.t.\ }}
\newcommand{\R}{\ensuremath{\mathbb{R}}}
\newcommand{\N}{\ensuremath{\mathbb{N}}}
\newcommand{\Z}{\ensuremath{\mathbb{Z}}}
\newcommand{\Q}{\ensuremath{\mathbb{Q}}}
\newcommand{\cl}{\ensuremath{\text{cl\hspace{0.1em}}}}

\newcommand{\card}{\ensuremath{\text{card\hspace{0.1em}}}}
\DeclareMathOperator{\blackboardE}{\mathbb{E}}
\newcommand{\expected}[1]{\blackboardE\left[#1\right]}
\newcommand{\expectedwhen}[2]{\blackboardE_{#2}\left[#1\right]}
\newlength{\postparenlength}
\setlength{\postparenlength}{0.14em}
\newcommand{\postparen}{\hspace{\postparenlength}}

\setlength{\parskip}{0.5\parindent}
\usepackage{url}
\usepackage{xcolor}

\usepackage[biblatex]{embrac}
\usepackage{hyphenat}
\doublehyphendemerits=20000  % Increase the penalty for multiple hyphenated lines in a row (note, in this case, >10000 isn't infinity)
\usepackage{siunitx}
\sisetup{
	group-separator={,},
	group-digits = integer,
	input-ignore = {,},
	input-decimal-markers = {.}
	}
\usepackage[authordate,isbn=false,backend=biber,autopunct=true,url=false]{biblatex-chicago}
%\DefineBibliographyExtras{american}{\stdpunctuation}
\renewcommand{\finalnamedelim}{\addspace \bibstring {and}\space}
%\renewcommand*{\bibfont}{\footnotesize}
%\DefineBibliographyStrings{english}{references = {References}}
%\setlength\bibitemsep{0pt}
\bibhang=\parindent
\addbibresource{./papers.bib}
\newcommand{\cex}{\textsc{cex}}
\newcommand{\apf}{\textsc{apf}}
\newcommand{\msrp}{\textsc{msrp}}
\newcommand{\eitc}{\textsc{eitc}}
\newcommand{\gdp}{\textsc{gdp}}
\newcommand{\vin}{\textsc{vin}}

\newcommand{\Var}{\text{Var}}

\usepackage{gitinfo2}  % requires git hooks!  See the Code/git_hooks folder in this repo.
\newcommand{\snippet}[1]{\input{./Generated_snippets/#1}\hspace{-0.15em}}
%\usepackage{footnote}
\usepackage{hyperref}
\hypersetup{colorlinks,
    linkcolor=black,
    filecolor=black,
    urlcolor=darkgray,
    citecolor=black,
    pdfpagemode=UseNone,
    pdftoolbar=false,
    pdftitle={Cars in Alaska},
    pdfauthor={Karl Dunkle Werner},
    pdfsubject={ARE second year paper},
    pdfcreator={},
    pdfproducer={},
    pdflang=en,
    unicode=true
}


\graphicspath{{./Plots/}, {./Plots/Daily/}}
\begin{document}
\thispagestyle{empty}
\setcounter{page}{0}
\vspace*{0.7in plus 0.3in minus 0.3in}

\begin{center}
    {\Huge Cars in Alaska}

    \href{mailto:karldw@berkeley.edu}{\textit{\Large Karl Dunkle Werner}}$^*$

\vspace{1em}
  This draft: \gitCommitterDate{} (version
    % The \gitHash here will automatically update the URL as necessary
    \href{https://github.com/karldw/second_year_paper/tree/\gitHash}{
    \textsc{\gitAbbrevHash{}}})\\
    \href{http://karldw.org/second_year_paper.pdf}{Click here for the most recent version.}
\end{center}

\vspace{2in plus 1in minus 0.7in}

\begin{center}
    \begin{minipage}{0.7\linewidth}
        \begin{center}
            \textit{Abstract}
        \end{center}
        Abstract text here \ldots
        \begin{itemize}
            \item Alaska payments are large
            \item I don't observe large effects, though some estimates are imprecise.
            \item Results are consistent with the permanent income hypothesis, dealers absorbing demand shocks with inventory or real\hyp{}but\hyp{}moderately\hyp{}sized effects
        \end{itemize}
    \end{minipage}
\end{center}

\vspace{3in plus 2in minus 1.7in}
{\footnotesize
\noindent
$^*$Many thanks to the Claire Duquennois, Elisa Duran Micco, Gabe Englander, Hal Gordon, Andy Hultgren, Stephen Jarvis, Scott Kaplan, Ben Krause, Megan Lang, Peiley Lau, Kate Pennington, Wenfeng Qiu, Eleanor Wiseman, Derek Wolfson, Jim Sallee and the faculty of the Energy Institute.
The numerous remaining errors are, of course, my own.
}

\pagebreak
%
% \textsc{\Large Outline}
%
% \begin{itemize}
%     \item Findings!
%     \begin{itemize}
%         \item Sales counts per capita
%         \item Sales volumes per capita
%         \item Mean efficiency (gallons per mile)
%         \item Vehicle value
%         \begin{itemize}
%             \item \msrp{}
%             \item Model year age
%             \item Auction price
%         \end{itemize}
%     \end{itemize}
%     \item Contribute to some literature strands:
%     \begin{itemize}
%         \item PIH
%         \item Fuel efficiency choice
%         \item IO of used auto markets
%     \end{itemize}
%     \item Papers to compare
%     \begin{itemize}
%         \item Hsieh
%         \item \textcite{Busse2015_weather_on_cars} about fickle car buying
%         \item Presentation about subsidy in EVs
%         \item \eitc{} literature (e.g.\ \cite{goodman2008eitc})
%     \end{itemize}
% \end{itemize}
%
% \pagebreak
\textsc{\Large Outline: draft for Feb 27}

\begin{itemize}
    \item Short paragraph on income effects, durable goods, why we care about people's car purchase decisions at all.
    \item Describe APF
    \item Contribute to some literature strands:
    \begin{itemize}
        \item PIH
        \item Fuel efficiency choice
        \item IO of used auto markets
    \end{itemize}
    \item Write theory of car buying, make predictions about results
    \begin{itemize}
        \item Explain why announcement date isn't going to be informative.
    \end{itemize}

    \item Findings -- small results
    \begin{itemize}
        \item Sales counts
        \begin{itemize}
            \item Very small coefficients, however you slice it.  Logs imply a somewhat noisier response.
            % anticipation_window_sale_count_weekly_notitle_coef_effect_mean.pdf
            % Then put these side-by-side:
            % variable_window_dd_sale_count_weekly_area_conf95_max_effect_mean.pdf
            % variable_window_dd_sale_count_log_weekly_area_conf95_max.pdf
        \end{itemize}

        \item Sales volumes
        \begin{itemize}
            \item  Only mention sales volumes in passing; the component pieces of sales\_count and sales\_pr are more interesting.
            \item The effects are small when estimated in levels, moderate with logs.
        \end{itemize}
        \item Mean efficiency (liters per 100~km)
        \begin{itemize}
            % TODO in next iteration: look at mechanisms: if we control for vehicle age, what happens to the effect? What about weight?
            \item Overall, not a large effect. We can rule out any effect larger than about 10\% of the mean, and the coefficients aren't statistically significant.
            % anticipation_window_fuel_cons_weekly_notitle_coef_effect_mean.pdf
            % Then put these side-by-side:
            % variable_window_dd_sale_count_weekly_area_conf95_max_effect_mean.pdf
            % variable_window_dd_sale_count_log_weekly_area_conf95_max.pdf

        \end{itemize}
        \item Vehicle value, as measured by one of the following:
        \begin{itemize}
            \item \msrp{}
            \begin{itemize}
                % TODO: note that the last week of MSRP has a huge coefficient, but that says more about some year-end effect than this check.  I've dropped it to make the graph easier to see.
                \item  For \msrp{} the conclusions are dramatically different for levels and logs.
                    For levels, the estimated effects are insignificant, and the confidence intervals rule out large effects -- anything larger than 10 or 20\% of the mean.
                    % variable_window_dd_msrp_mean_weekly_area_conf95_max_effect_mean.pdf
                    For logs, the point estimates and confidence intervals allow for much larger results.
                    With point estimates around 0.35 and confidence interval upper bounds around 0.7 (for reasonable specifications; not those starting at the beginning), we can't reject large effects.
                    At the same time, the lower bound of the CIs is \emph{just} above zero, so I don't want to put too much stock in the statistical significance shown in the plot.
                    % variable_window_dd_msrp_mean_log_weekly_area_conf95_max.pdf
            \end{itemize}
            % \item Model-year age (TODO: add vehicle age in next draft)
            \item Auction price (valid as a unified measure of vehicle quality if there aren't market failures)
            \begin{itemize}
                \item  As with \textsc{msrp}, there's a difference between logs and levels here.  With levels, the results are insignificant, the point estimates are very close to zero and we can reject most results above 15\% of the mean price.  (The mean is \$14,912.13.)
                With logs, the point estimates again imply larger effects, around 0.3, with confidence intervals around 0.7.
                % Show two pairs here:
                % Pair a:
                % effects_individual_period_sales_pr_mean_weekly_coef_notitle.pdf
                % effects_individual_period_sales_pr_mean_log_weekly_coef_notitle.pdf
                % Pair b:
                % variable_window_dd_sales_pr_mean_weekly_area_conf95_max_effect_mean.pdf
                % variable_window_dd_sales_pr_mean_log_weekly_area_conf95_max.pdf
            \end{itemize}
        \end{itemize}
        \item Show that results are robust to dropping any particular year (make a graph)
        \item Acknowledge the issues of multiple testing -- they cut both ways, with my ``precise'' zeros being somewhat less precise.
    \end{itemize}


    \item Papers to compare
    \begin{itemize}
        \item Hsieh
        \item \textcite{Busse2015_weather_on_cars} about fickle car buying
        \item Presentation about subsidy in EVs
        \item \eitc{} literature (e.g.\ \cite{goodman2008eitc})
    \end{itemize}
\end{itemize}

% {\Large Questions for Jim}
% \begin{itemize}
%     \item What papers should I cite when I mention the energy efficiency gap?
%     \item Am I interpreting the log analysis right?  It seems odd that the confidence intervals there would be so large.
% \end{itemize}
\pagebreak
\setcounter{page}{1}
\begin{doublespacing}
\section{Introduction}

Automobile sales make up a large portion of the US economy, generate significant externalities and are some of the largest purchases consumers make.
For these reasons, we should be interested in what factors drive automobile purchases.
Since vehicles are substantial investments, it's natural to expect income effects -- if a consumer's budget constraint is relaxed (exogenously and without anticipation), their purchase decisions should change.
The  Alaska Permanent Fund provides an interesting source of income variation, allowing me to look at a few aspects of the income effect in car purchases.
In the results below, I find statistically insignificant and economically small effects for most results.

The Alaska Permanent Fund is an account set up to hold and invest a portion of the state's mineral rights revenue.
Since the 1980s, the fund has been sending dividend payments in the fourth quarter of each year to Alaskan residents \parencite{hsieh2003}.
These dividends are substantial, roughly \$1000--2000 per person, as shown in figure~\ref{fig:permanent-fund-payments-individual} \parencite{apfd_payments_summary}.
In the figure and for the rest of this article, all dollar amounts are in 2016 dollars, unless indicated otherwise.
The dividend amount is not contingent on age or income, so a large family may receive more than \$10,000.
In recent years, the payment has happened on the first Thursday in October.
I will use this somewhat\hyp{}arbitrary payment date as a source of variation in household's income, and therefore willingness to buy vehicles.
% It's worth mentioning a few of the dividend details before going on.
The dividend is paid out of investment earnings -- 10.5\% of the past five years' earnings -- not out of present\hyp{}day oil revenue.
The Permanent Fund investments are made in a broad variety of stocks and bonds, and the dividend amount is very predictable.
\footnote{The asset allocation is detailed in the Alaska Permanent Fund Corporation's website, \url{http://apfc.org/home/Content/investments/assetAllocation2009.cfm}}
The dividend amount is finalized and announced a few weeks before payment, and Alaskan newspapers publish accurate predictions before that \parencite{adn_dividend_prediction, adn_dividend_realization}.

\begin{figure}[bht]
    \caption{\large Alaska Permanent Fund payments per person}
    \label{fig:permanent-fund-payments-individual}
    \includegraphics[width=\linewidth]{permanent_fund_payments_individual_notitle.pdf}

\noindent\textsc{Notes:}
		Figures are adjusted for inflation using the annual average \textsc{cpi} \parencite{fred_inflation}.
		The 2008 total includes a one-time bonus of \$1200 (nominal dollars).
\end{figure}


The dividend the same amount for all recipients, with no means testing or other adjustments and almost everyone in Alaska gets the dividend.
Major exceptions are people who have lived in the state for less than a year and people serving time for some crimes.
Approximately 91\% of the state's population applies and about 95\% of those applications are granted.


The principal data source in my analysis is the Manheim wholesale used vehicle auction data.
The dataset records sale prices, vehicle identifiers, buyer and seller IDs and locations.
After various cleaning measures described in section~\ref{sec:manheim-data}, I have
\snippet{auctions_cleaned_total_obs_count.tex}
observations of vehicle sales,
\snippet{auctions_cleaned_alaska_obs_count.tex} in Alaska.
These sales are between wholesalers such as auto dealerships, rental car agencies and vehicle leasing programs.
In the following analysis, I focus on wholesalers with billing zip codes in Alaska who buy vehicles in the auctions, and for simiplicity, I call all wholesalers ``dealers.''

Manheim auctions occur in 32 states, but there are no auction sites in Alaska, so all of the purchases I observe for these dealers are from out of state.
While I see when the vehicles were bought at auction, I don't know when they were delivered to the dealer or sold to the final customer.
Presumably, there can be some substantial delay shipping to Alaska; I examine the delay and anticipation effects as part of my analysis.

Importantly, I don't observe final sales to customers, so I don't know the timing or their purchases or the prices they are charged.
I assume that cars bought by Alaskan dealers are sold to consumers with a not\hyp{}to\hyp{}long delay -- since it's costly to hold unsold cars on the lot -- and not resold to another wholesaler.
To minimize the resale observations, I exclude cars that are bought and resold within a year, though this only addresses resales where both the initial sale and the second are in the auction data.
The Manheim data include \emph{used} vehicles only, so I will only be able to discuss the market for new vehicles in my theoretical speculation (section~\ref{sec:model}).
% The Manheim data are described in more detail in section~\ref{sec:manheim-data}.


% \subsection{Literature}
I mentioned that a large, \emph{unanticipated} income shock may have substantial effects.
The permanent income hypothesis literature emphasizes the importance of anticipation.
Since the Alaskan dividend timing and amount are not surprises, the strong form of the hypothesis would predict no response at all.
The permanent income hypothesis would also predict an increase in consumption when the payment is \emph{announced}, but because the dividend date has been well known many years, I'm not able to examine the consumption decisions around a specific announcement date.
In the real world, liquidity constraints, uncertainty and behavioral factors may mean that people don't smooth their consumption perfectly.

The previous paper analyzing the Alaska Permanent Fund dividends, \textcite{hsieh2003}, used the consumer expenditure survey to look at changes in consumption from the third quarter to the fourth.
Hsieh also didn't reject the consumption smoothing hypothesis, though his identifying variation relied on potentially\hyp{}endogenous differences in family size and year\hyp{}to\hyp{}year dividend variation.

I'm unaware of other papers looking at the consumption effects of the Alaskan dividend, but many researchers have looked at tax returns.
For example, \textcite{goodman2008eitc} look at \eitc{} payments and find that \eitc{} recipients seem to spend their refunds on automobiles.

In addition to consumption smoothing, we may be interested in durable goods purchases because those goods determine energy use.
Some authors have argued that consumers undervalue fuel efficiency, and if so, it's possible that having more cash on hand would lead a consumer to make a longer\hyp{}sighted decision and include the full value of fuel efficiency savings.
On the other hand, more recent empirical work has found that consumers' vehicle choices include most or all of the value of saved fuel \parencite{busse2013consumers, allcott2014gasoline, grigolon2014consumer, sallee2016consumers}.



\section{Simple Model}
\label{sec:model}

% TODO First explain in words, then make formal in math.  Figure out good, non-conflicting syntax.

\subsection{Consumers}


People value the stream of services provided by owning a vehicle, ad so they have preferences for vehicle traits and vehicle ownership.
Consumers are constrained by their budgets, so they may therefore change what they buy when they have a check from the government.
Even though the check is anticipated, they may not fully smooth because they don't build up enough savings in the course of their normal consumption path (if, for example, they act as buffer\hyp{}stock savers).

There are also behavioral possibilities, both in lack of smoothing and people treating dividend checks differently than they would treat their own savings.
I won't really be able to disentangle these, since I don't have any situations where saved wealth is exogenously manipulated, but it's important to consider that responses may differ between this situation and others.

For simplicity, I'll assume each buyer purchases at most one car, indexed by $v$.
I'll also assume each vehicle's life ($L_v$) is known with certainty, and that there are no maintenance costs.
Let $\mathcal{V}_u$ and $\mathcal{V}_n$ be the set of available used and new vehicles, and $\mathcal{V}_0$ be a non\hyp{}purchase, so the consumer's full choice set is $\mathcal{V} \equiv \mathcal{V}_u \cup \mathcal{V}_n \cup \mathcal{V}_0$.
Let $g_v$ be the (endogenous) discounted gasoline expenditures associated with vehicle $v$, and $p_v$ the retail price of the vehicle.
$w$ is the wealth, including Alaskan dividends, of the consumer, so with perfect borrowing and lending markets, the net wealth is $w - g_v - p_v$.
The consumer then chooses $v$ to maximize the integral of instantaneous utility:
\[
\max_{v \in \mathcal{V}} \ \ \int_0^{L_v} u(v, w - p_v - g_v) e^{-rt} dt
\]

\subsection{Auto dealers}

Economic intuition says that car dealers would not be thrilled to hold onto vehicles for longer than necessary, if only because of the opportunity cost of capital.
For instance, US News offers the advice, ``If the car has been there for more than three months, the dealer will be more anxious to sell it as quickly as possible''
\parencite{usnews_car_deals}.

To make the model less complex, I'll assume auto\hyp{}dealers don't exercise market power when they buy their used cars, that they know consumers' preferences, and that it's costly to hold inventory.
I don't exclude market power on the retail side, and since I don't have data on retail sales.

Assuming away market power is clearly a big assumption, but Alaska is a small part of the overall used car market, and making it unlikely that individual Alaskan buyers act as monopsonists in the auction market.

With undifferentiated products, prices $p$, quantity $q(p)$, unit cost $c$, time-to-sell $T(p)$ and interest rate $r$, the dealer's problem is:
\[
\max_p \ \ p \cdot q(p) - c \int_o^{T(p)} e^{-rt}dt
\]

With differentiated products, the problem becomes harder.
Assuming consumers buy more used cars upon receiving their dividend checks, dealers adjust.
Specifically, the dealers know what kinds of cars will be popular and will buy these beforehand, making sure they're available when the dividends arrive.
To consider the problem analytically, a dealer would consider the cross-price elasticities of demand and time-to-sell for different vehicles.
It seems likely that  the dividend checks influence both new and used car purchases, as outlined in the consumer model.
Some dealers specialize in used vehicles, while others sell both used and new vehicles.




\section{Data}
\label{sec:data}

\subsection{Manheim wholesale auto auctions}
\label{sec:manheim-data}



% TODO: Repeat a little about the used car data and the things I can observe. Make clear again that I can't see sales to final consumers.
The Manhiem auctions dataset includes a rich set of variables; I've only used the more basic ones.
The data include identifiers for buyers, sellers and auction sites, as well as the billing zipcodes for all three.
I use the zipcodes to identify which wholesalers are in Alaska or one of my control states.

\begin{table}[hbt]
    \caption{Cleaning Manheim auction data}
    \label{tab:cleaning_manheim}
    % TODO: Make the numbers here better by having parse-numbers = true (currently broken because the snippet gets added after siunitx tries to parse the number)
    \sisetup{parse-numbers = true}
\begin{tabular}{l S p{0.493\linewidth}}
    \toprule
	Elimination category & \multicolumn{1}{l}{Count removed} & Details\\
	\midrule

    Duplicate sales &
    % \snippet{clean_car_auctions_resold.tex}
    3551583
    & Same \textsc{vin} sold twice within a year.\\

    \addlinespace

    Weird vehicles &
    % \snippet{clean_car_auctions_weird_vehicles.tex}
    20257
    & Trailers, boats, air compressors, golf carts, vehicles with incomplete bodies, \textsc{atv}s and \textsc{rv}s.\\

	\addlinespace

    Bad odometer &
    % \snippet{clean_car_auctions_bad_odometer.tex}
    563074
    & Auction comments indicate odometer is flawed.\\

	\addlinespace

    Damaged &
    % \snippet{clean_car_auctions_damaged.tex}
    2520692
    & Auction comments indicate vehicle is damaged. \\

	\addlinespace

    Bad price &
    % \snippet{clean_car_auctions_price.tex}
    59251
    & Auction price seems unreasonable, outside the interval $[100, \min\{80000, 1.5\times \textsc{msrp} \}]$.\\

	\addlinespace

    Canadian &
    % \snippet{clean_car_auctions_canadian.tex}
    121965
    & Auction comments indicate vehicle is Canadian.\\

    \bottomrule
    \addlinespace
\end{tabular}
\footnotesize
\textsc{Notes:} Data are from US Manheim auctions, 2002--2014.
There are \snippet{auctions_uncleaned_total_obs_count.tex} sales before cleaning and \snippet{auctions_cleaned_total_obs_count.tex} sales after.
\snippet{auctions_cleaned_alaska_obs_count.tex} of the cleaned sales are to Alaskan buyers.
Cars that are listed for auction but not sold are not included.
Rows are removed in the order listed.
\end{table}

The Manheim data are voluminous, and they include many kinds of sales outside the scope of this research.
In table~\ref{tab:cleaning_manheim}, I detail the cleaning I've done and the observation count removed in each.
When I observe the same vehicle (same 17-digit \vin{}) sold two or more times in a one-year period, I only use the last sale.
Some items in the auction are a bit odd, such as trailers, boats, air compressors, golf carts, vehicles with incomplete bodies, \textsc{atv}s and \textsc{rv}s.
I exclude all of these, keeping only cars and trucks.
The Manheim data include a variable for the vehicle's odometer, as well as a field of the auction comments.
In some cases, the comments indicate that the odometer is wrong, either because it's broken or because it is a five\hyp{}digit odometer that can't record more than 100,000 miles.
I drop these as well because I'm eventually planning to include the odometer in my analysis.
Many vehicles are designated as damaged, some very severely.  I drop these, since they're often not in a condition to be sold as a retail vehicle.
Some of the prices in the data seem unreasonable, either too high or too low, possibly indicating an error in data entry or something strange with that particular auction.
I drop cars sold for less than \$100 (in 2016 dollars) and cars sold for more than \$80,000 or 150\% of their $\textsc{msrp}$.
Finally, I drop cars that are flagged as Canadian, since these are presumably in a different market than US cars.
(The odometer may be measured in kilometers, there may be product differences and so on.)
Of these, only the duplicate sales and damaged vehicles eliminate a significant number of sales.


\subsection{VIN decoder}

Every modern vehicle has a unique 17-digit vehicle identification number (\vin{}).
The ID is broken in to pieces that identify the country, manufacturer, vehicle, model, trim, year and plant, plus a sequential identifier \parencite{vin_details}.
I have a database of vehicle characteristics at the country\hyp{}manufacturer\hyp{}vehicle\hyp{}model\hyp{}trim\hyp{}year level, which I match with the auction data to find the vehicles' suggested retail price (\msrp{}) and fuel economy.
% TODO: how many aren't matched?
% TODO: add N to the regression plots.

For fuel economy I use the \textsc{epa} combined rating -- a weighted average of city and highway miles per gallon -- and convert to liters per 100~km.
I do this because fuel consumption is generally much better behaved than fuel efficiency; the reciprocal can lead to unintuitive or poorly behaved estimates.
Rather than simply taking the reciprocal and having gallons per mile, I convert to liters per 100~km because no one uses gallons per mile, while liters per 100~km is at least interpretable by people who have lived outside the US\@.

\subsection{Alaskan dividends}
Note: the amounts don't matter because I have year FE



\section[DD Results]{Difference-in-Differences Results}

The auction data have before- and after\hyp{}dividend data for treated and untreated states, so a difference\hyp{}in\hyp{}differences framework is a natural framing.
Alaska is different from many other states, so picking appropriate controls is an important and somewhat subjective endeavor.
Below, in section~\ref{sec:picking-controls} I discuss how I've chosen the current set of controls: Utah, Idaho and Montana.
In section~\ref{sec:synthetic-controls} I discuss next steps to formalize the selection of control states.

For the difference\hyp{}in\hyp{}differences analysis, I've chosen a default window of 70~days (10~weeks) before and after the early\hyp{}October dividend.
This fairly wide window allows for various anticipation or delayed effects without getting too close to the end of the year and the associated sales slowdown.


\subsection{Picking control states}
\label{sec:picking-controls}
In considering which states are good controls for Alaska, it's worth diving a little more into the structure of the Manheim auction data.
Auctions occur in 32 states and Puerto Rico, and Alaska is not among them.
Therefore all purchases by Alaskan buyers are at out\hyp{}of\hyp{}state auctions.
For the simple difference\hyp{}in\hyp{}differences below, I've chosen Idaho, Oregon and Utah.
The choice is somewhat subjective -- I chose these states because they have a large fraction of their out\hyp{}of\hyp{}state purchases in the auction states where Alaskan dealers are buying.
This heuristic is appealing because it captures the markets where Alaska is operating, as well as additional factors, such as transportation costs, that arise from dealers buying their vehicles out of state.
% See the code in find_match_states_crude_unmemoized for full details..
A more robust approach is to use best\hyp{}subset or synthetic controls methods described in \textcite{DoudchenkoImbens2016DD} and section~\ref{sec:synthetic-controls}, and I plan to implement these in a future iteration of this project.

\begin{figure}[bth]
	\caption{Alaska-only and pooled mean and standard deviations}
	\label{fig:alaska_vs_pooled_mean_sd}
	\includegraphics[width=\linewidth]{comparison_alaska_vs_pooled_mean_sd.pdf}
\end{figure}

In addition to worrying about the control group for difference\hyp{}in\hyp{}differences, one might also worry about the effect size measurements.
In some of the analyses below, I divide by the sample mean or standard deviation to look at the size of the estimated coefficient.
Figure~\ref{fig:alaska_vs_pooled_mean_sd} plots out the mean and standard deviation I use for these normalizations.
Using the event window sample (70 days before to 70 days after the dividend) is almost identical to the full\hyp{}year sample, but Alaska is a smaller market than the controls I've chosen.
Alaska has a lower mean and standard deviation in cars sold, and therefore the total value of sales.
As well, the cars sold in Alaska are a bit more expensive and a bit less efficient.
It's worth keeping these differences in mind in the outcomes below.
The log specifications adjust for percentage differences more directly.

\subsection{Picking an anticipation window}

As mentioned in section~\ref{sec:model}, we might expect consumers to sharply change their consumption when they receive the dividend checks.
Auto dealers would anticipate that behavior and change their auction purchases somewhat earlier, depending on transportation time and the costs of holding inventory.

On the other hand, dealers anecdotally say they just try to maintain their inventories.
While the dealers know that customers tend to  buy more cars around the dividend date and when they receive tax refunds, but the dealers report that don't make additional efforts to stock up before the dividend is sent.
In this case, the auction data should show no anticipation effect.
Instead, there should be some increase in auction sales after the dividend, as customers buy cars and deplete dealers' inventories, which the dealers then restock.

% \begin{figure}[hbt]
%     \caption{Anticipation effects for sales count}
%     \label{fig:anticipation_window_sale_count}
%     \includegraphics[width=\linewidth]{anticipation_window_sale_count_weekly_notitle_coef_effect_mean.pdf}
%
%     {\footnotesize
%     The plot shows the Alaska $\times$ anticipation coefficients ($\beta_2$) from the regression in equation~\ref{eqn:cars_sold_with_anticipation}, with varying anticipation windows.
%     Error bars indicate 95\% confidence intervals.
%     %The horizontal line at \snippet{sales_count_weekly_std_dev.tex} is the standard deviation of weekly counts.
%     All windows end at $-1$ week.
%     }

% \end{figure}
% \begin{figure}[hbt]
%      \caption{Anticipation effects for sales totals}
%      \label{fig:anticipation_window_sale_tot}
%         \includegraphics[width=\linewidth]{anticipation_window_sale_tot_weekly_notitle_coef_effect_mean.pdf}
%
%     {\footnotesize
%     The plot shows the Alaska $\times$ anticipation coefficients ($\gamma_2$) from the regression in equation~\ref{eqn:cars_sold_with_anticipation}, with varying anticipation windows.
%     Error bars indicate 95\% confidence intervals.
%     %The horizontal line at \snippet{sales_tot_weekly_thousands_std_dev.tex} is the standard deviation of weekly sales volume (in thousands).
%     All dollar amounts are in thousands of 2016 dollars.
%     All windows end at $-1$ week.
%     }
% \end{figure}


Anecdata are often wrong, so maybe it's worth considering the theoretical predicting a dealer anticipation effect.
Theory doesn't tell us, however, how long that anticipation window should be.
To be as flexible as possible, I have estimated the results for varying windows from one to nine weeks of anticipation, as shown in figures~\ref{fig:anticipation_window_sale_count} and~\ref{fig:anticipation_window_sale_tot}.
% TODO: remove the previous figure reference. (Replace with something else?)
In all cases, the anticipation window ends the immediately before the dividend day.

None of the estimates are statistically significant, and they are all far below the standard deviation of weekly sales counts and sales totals, as indicated by the horizontal lines at the top of the graphs.
Different states have somewhat different standard deviations, the pooled standard deviation is not driven by states with particularly large variation.
The 95\% confidence interval of the anticipation estimates less than are standard deviation for Alaska and every control state; figure~\ref{fig:alaska_vs_pooled_mean_sd} plots the pooled and Alaska\hyp{}only mean and standard deviation.
I default to calculating the pooled mean across Alaska and the control states and across all weeks in the data.
Calculating the meanon only periods within the event window -- from 70~days before the dividend is sent to 70~days after -- gives very similar results as the all\hyp{}year sample.

Using daily data produces similar results.
The coefficients are somewhat more variable and a couple are statistically significant, but they're still much smaller than the standard deviation.
% (See figure~\ref{bonusfig:anticipation_window_sale_count_daily}.)


% \begin{figure}[hbt]
%      \caption{Sales totals effect size across windows, max 95\% CI}
%      \label{fig:variable_window_sale_tot_conf95_max_effect}
%         \includegraphics[width=\linewidth]{variable_window_dd_sale_tot_weekly_area_conf95_max_effect.pdf}
%
%     {\footnotesize
%     % TODO: write an equation for this, put in the appropriate greek letter. (use \psi or change the math below)
%     \textsc{Notes:} The plot shows the Alaska $\times$ anticipation coefficients on total sales from a regression with a varying window.
%     The $y$-axis shows is the start of the window relative to the dividend day, while the $x$-axis codes the end of the window.
%     The size of the points is the maximum of the absolute value of the effect size of the 95\% confidence intervals: $\max \big\{ \hat{\psi} - 1.96 \sqrt{\Var(\hat{\psi})},$ $ \hat{\psi} + 1.96 \sqrt{\Var(\hat{\psi})} \big\}$.
%     A standard DD regression -- where the window begins at zero and ends at the end of the period -- is the rightmost point on the $y=0$ line.
%     }
% \end{figure}

% \begin{figure}[hbt]
%      \caption{Maximum sales count effect size confidence intervals}
%      \label{fig:variable_window_sale_count_conf95_max_effect}
%         \includegraphics[width=\linewidth]{variable_window_dd_sale_count_weekly_area_conf95_max_effect.pdf}
%
%     {\footnotesize
%     % TODO: write an equation for this, put in the appropriate greek letter. (use \psi or change the math below)
%     \textsc{Notes:} The plot shows the Alaska $\times$ anticipation coefficients on sale count from a regression with a varying  window.
%     The $y$-axis shows is the start of the window relative to the dividend day, while the $x$-axis codes the end of the window.
%     The size of the points is the maximum of the absolute value of the effect size of the 95\% confidence intervals: $\max \big\{ \hat{\psi} - 1.96 \sqrt{\Var(\hat{\psi})},$ $ \hat{\psi} + 1.96 \sqrt{\Var(\hat{\psi})} \big\}$.
%     A standard DD regression -- where the window begins at zero and ends at the end of the period -- is the rightmost point on the $y=0$ line.
%     }
% \end{figure}

On the other hand, one might put more credence in the anecdotes.
To show that there is still no effect on vehicle sales, I'll estimate an even more flexible treatment window.




% TODO: in addition to comparing vs the std dev, also compare as a pct of the mean.

% \begin{figure}[hbt]
%     \caption{Treatment coefficients on Alaska $\times$ anticipation: total sales}
%     \label{fig:anticipation_window_sale_tot}
%     \includegraphics[width=\linewidth]{anticipation_window_sale_tot_weekly_notitle_pooled_sd.pdf}
%
%     {\footnotesize
%     The plot shows the Alaska $\times$ anticipation coefficients ($\gamma_2$) from the regression in equation~\ref{eqn:cars_sold_with_anticipation}, with varying anticipation windows.
%     Error bars indicate 95\% confidence intervals.
%     The horizontal line at \$\snippet{sales_tot_weekly_thousands_std_dev.tex} is the standard deviation of weekly sales volume (in thousands).
%     All windows end at $-1$ week.
%     }
% \end{figure}

\subsection{Volume of cars sold}
% TODO: move this section up so I can talk about number and volumes of cars sold
\begin{align}
    \label{eqn:cars_sold_with_anticipation}
    \text{Cars sold} = & \ \beta_1 \ \text{anticipation} + \beta_2 \ \text{Alaska} \times \text{anticipation} + \beta_3 \ \text{post-dividend} \\ \nonumber
    &+ \beta_4 \ \text{Alaska} \times \text{post-dividend} + \text{state and year fixed effects}
    %\\
    % \text{Sale volume} = & \ \gamma_1 \ \text{anticipation} + \gamma_2 \ \text{Alaska} \times \text{anticipation} + \gamma_3 \ \text{post-dividend} \\ \nonumber
    % &+ \gamma_4 \ \text{Alaska} \times \text{post-dividend} + \text{state and year fixed effects}
\end{align}

% Collapse data to a state-by-day panel and run DD regs of sale count per capita and sale volume per capita.
% Use all years, use event time relative to the dividend day and don't take into account any anticipation.
% Control for either a year FE or state \gdp{}. Control for either daily or day-of-week FE.
% As a default, the cars sold in the set of states where Alaskan buyers are active (in a given year).
% Force the panel to be balanced by adding zeros (if necessary).

This is theoretically ambiguous -- sales of \emph{used} cars could go up or down with an income shock, or be unchanged.


\begin{figure}[hbt]
    \caption{Treatment coefficients on Alaska $\times$ anticipation}
    \label{fig:anticipation_window_sale_count}
    \includegraphics[width=\linewidth]{anticipation_window_sale_count_weekly_notitle_coef_effect_mean.pdf}

    {\footnotesize
    \textsc{Notes:}
    % The plot shows the Alaska $\times$ anticipation coefficients ($\beta_2$) from the regression in equation~\ref{eqn:cars_sold_with_anticipation}, with varying anticipation windows.
    All windows end at $-1$ event weeks.
    }
\end{figure}

\begin{figure}[hbt]
    \caption{Sales count effects with varying windows}
    % \label{bonusfig:anticipation_window_sale_count_daily}
    \begin{subfigure}{\linewidth}
        \caption{Estimate in levels, scaled by weekly mean sales}
        \includegraphics[width=\linewidth]{variable_window_dd_sale_count_weekly_area_conf95_max_effect_mean.pdf}
    \end{subfigure}
    \begin{subfigure}{\linewidth}
        \caption{Estimate in logs}
        \includegraphics[width=\linewidth]{variable_window_dd_sale_count_log_weekly_area_conf95_max.pdf}
    \end{subfigure}


    {\footnotesize
    \textsc{Notes:}
    Each dot is one regression.
    The $y$- and $x$-axes represent the window start end, measured in weeks before or after the dividend day
    The dividend day starts week zero.

     The area of the circles represent the magnitude of the larger bound of the 95\% confidence interval for each regression.
    In plot (a), the values are scaled by the sample mean weekly sale count.
    (Using the pooled mean across Alaska and control states; see figure~\ref{fig:alaska_vs_pooled_mean_sd} for details.)
    Sign is the sign of the coefficient estimate, and because confidence intervals are symmetric, the sign of the larger bound of the 95\% confidence interval.
    }
\end{figure}


In addition to the number of cars sold, we could look at total sales volume in dollars.
I find the component pieces -- the number of cars and the price of each -- more interesting, so I won't linger on the total sales volume analysis.
As with sales count, the effects are quite small when estimated in levels, with wider confidence intervals when estimated in logs.

\subsection{Quality of cars sold}

\begin{figure}[hbt]
    \caption{MSRP effects with varying windows}  % Not \msrp here
    % \label{}
    \begin{subfigure}{\linewidth}
        \caption{Estimate in levels, scaled by weekly mean MSRP}
        \includegraphics[width=\linewidth]{variable_window_dd_msrp_mean_weekly_area_conf95_max_effect_mean.pdf}
    \end{subfigure}
    \begin{subfigure}{\linewidth}
        \caption{Estimate in logs}
        \includegraphics[width=\linewidth]{variable_window_dd_msrp_mean_log_weekly_area_conf95_max.pdf}
    \end{subfigure}


    {\footnotesize
    \textsc{Notes:}
    Each dot is one regression.
    The $y$- and $x$-axes represent the window start end, measured in weeks before or after the dividend day
    The dividend day starts week zero.

     The area of the circles represent the magnitude of the larger bound of the 95\% confidence interval for each regression.
    In plot (a), the values are scaled by the sample mean weekly \msrp{}.
    (Using the pooled mean across Alaska and control states; see figure~\ref{fig:alaska_vs_pooled_mean_sd} for details.)
    Sign is the sign of the coefficient estimate, and because confidence intervals are symmetric, the sign of the larger bound of the 95\% confidence interval.
    }
\end{figure}


For \msrp{} the conclusions are dramatically different for levels and logs.
    For levels, the estimated effects are insignificant, and the confidence intervals rule out large effects -- anything larger than 10 or 20\% of the mean.

    For logs, the point estimates and confidence intervals allow for much larger results.
    With point estimates around 0.35 and confidence interval upper bounds around 0.7 (for reasonable specifications; not those starting at the beginning), we can't reject large effects.
    At the same time, the lower bound of the CIs is \emph{just} above zero, so I don't want to put too much stock in the statistical significance shown in the plot.


\subsection{Efficiency of cars sold}

Why look at efficiency?
Some people have speculated that there's an energy efficiency gap, where people are failing to appreciate the energy efficiency implications of their purchases.
If that's true, it may be that the dividend check provides liquidity and allows the consumer to more optimally trade off current and future costs.
More recent literature has found that people buy incorporate much \parencite{allcott2014gasoline} or all \parencite{busse2013consumers, grigolon2014consumer, sallee2016consumers} of the expected discounted fuel expenditures in their vehicle purchase decisions.

As \textcite{kiso2013automobilefueleconomy} reminds us, fuel economy is intimately related to other vehicle characteristics, so even if consumers are already pricing in fuel costs, having more money to spend on a vehicle may change the fuel economy.
Since burning fuel generates externalities, the question still matters,

In the difference\hyp{}in\hyp{}differences analysis, the effect is rather small.
I can rule out any effect larger than about 10\% of the mean, and the coefficients aren't statistically significant.




\begin{figure}[hbt]
    \caption{Fuel consumption treatment coefficients on Alaska $\times$ anticipation}
    % \label{bonusfig:anticipation_window_sale_count_daily}
    \includegraphics[width=\linewidth]{anticipation_window_fuel_cons_weekly_notitle_coef_effect_mean.pdf}

    {\footnotesize
    \textsc{Notes:}
    % The plot shows the Alaska $\times$ anticipation coefficients ($\beta_2$) from the regression in equation~\ref{eqn:cars_sold_with_anticipation}, with varying anticipation windows.
    All windows end at $-1$ event weeks.
    }
\end{figure}


\begin{figure}[hbt]
    \caption{Fuel consumption effects with varying windows}
    % \label{bonusfig:anticipation_window_sale_count_daily}
    \begin{subfigure}{\linewidth}
        \caption{Estimate in levels, scaled by weekly mean sales}
        \includegraphics[width=\linewidth]{variable_window_dd_fuel_cons_weekly_area_conf95_max_effect_mean.pdf}
    \end{subfigure}
    \begin{subfigure}{\linewidth}
        \caption{Estimate in logs}
        \includegraphics[width=\linewidth]{variable_window_dd_fuel_cons_log_weekly_area_conf95_max.pdf}
    \end{subfigure}


    {\footnotesize
    \textsc{Notes:}
    Each dot is one regression.
    The $y$- and $x$-axes represent the window start end, measured in weeks before or after the dividend day
    The dividend day starts week zero.

     The area of the circles represent the magnitude of the larger bound of the 95\% confidence interval for each regression.
    In plot (a), the values are scaled by the sample mean weekly fuel consumption.
    (Using the pooled mean across Alaska and control states; see figure~\ref{fig:alaska_vs_pooled_mean_sd} for details.)
    Sign is the sign of the coefficient estimate, and because confidence intervals are symmetric, the sign of the larger bound of the 95\% confidence interval.
    }
\end{figure}



\section{Planned Extensions}

\subsection{Synthetic controls}
\label{sec:synthetic-controls}
Do the previous DD section again, but with synthetic controls or, maybe, the fancier frameworks of
\textcite{DoudchenkoImbens2016DD} or \textcite{Xu2016}

\subsection{Standard errors}
% Elaborate:
The standard errors are only robust to heteroskedasticity, not any form of autocorrelation, so the \textcite{bertrand_duflo2004DD} critique applies.
I haven't clustered because I don't have enough clusters for the asymptotics to be valid.

\subsection{Other measures of vehicle quality}
\begin{itemize}
    \item Age
    \item Depreciated value
\end{itemize}

\subsection{Consumer side}
\subsubsection{Quarterly Polk data}
% TODO: run a regression with Hsieh-style quarter-over-quarter registration counts
\subsubsection{Other registration data}

\subsection{Robustness}
Not all years are the same.
A natural robustness check is to exclude one year at a time and see if the results change dramatically.

\subsection{Auto market IO}
\begin{itemize}
    \item Buyer-dealer interactions and market power
    \item Repeating the while analysis at the buyer\hyp{}week or buyer category\hyp{}week level.
    \item Price effects -- dealers in Alaska paying more for the same \vin10 in years with high dividends.
    % To get that right, it's important to control for vehicle quality carefully; a higher price could indicate market shifts, or just that Alaskan dealers are shifting to newer used cars.
    % (Both are interesting, but they're different stories.)

\end{itemize}



\section{Conclusion}

\begin{itemize}
    \item Generally, not huge effects.
    \item Mention the outcomes specifically.
    \item Why would that be?
    \begin{itemize}
        \item Possible that people aren't responding in a dramatic way
        \item Possible that consumers \emph{are} responding dramatically, but the shifts are absorbed in inventory, which the dealers gradually replenish.

    \end{itemize}
\end{itemize}

\end{doublespacing}
\FloatBarrier
\pagebreak
% References
\printbibliography

\begin{refsection}[software.bib]
\nocite{*}
\printbibliography[heading=subbibliography, title={Software Used}]
\end{refsection}


\end{document}
\pagebreak


\appendix

\section{Bonus Figures for Jim}

\begin{figure}[hbt]
    \caption{Treatment coefficients on Alaska $\times$ anticipation}
    \label{bonusfig:anticipation_window_sale_count_daily}
    \includegraphics[width=\linewidth]{anticipation_window_sale_count_daily_notitle.pdf}
    {\footnotesize
    The plot shows the Alaska $\times$ anticipation coefficients ($\beta_2$) from the regression in equation~\ref{eqn:cars_sold_with_anticipation}, with varying anticipation windows.
    All windows end at $-1$ event day.
    }
\end{figure}

\begin{figure}[hbt]
	\caption{Treatment coefficients on Alaska $\times$ anticipation, with state-specific standard deviation lines}
	\label{bonusfig:anticipation_window_sale_count_and_tot_weekly_states_sd}
	\includegraphics[width=\linewidth]{anticipation_window_sale_count_weekly_notitle_states_sd.pdf}

	\includegraphics[width=\linewidth]{anticipation_window_sale_tot_weekly_notitle_states_sd.pdf}
\end{figure}



\end{document}
