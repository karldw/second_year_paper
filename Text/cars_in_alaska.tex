\documentclass[11pt,letterpaper,oneside]{article}
\usepackage[utf8]{inputenc}
\usepackage[T1]{fontenc}
\usepackage{amssymb,amsmath,amsthm}
\usepackage{enumitem}
\usepackage[mathlf]{MinionPro}
\usepackage{carlito}
% \usepackage{roboto}
\usepackage{inconsolata}
\usepackage{sectsty}
\usepackage{setspace}
% \onehalfspace
\doublespace
\usepackage[font=sf]{caption}
\allsectionsfont{\sffamily\mdseries}
\usepackage[activate={true,nocompatibility},final,tracking=true,spacing=true]{microtype}
\microtypecontext{spacing=nonfrench}
\usepackage{typearea}   % Let the typearea package work out the best margins (wider than the defaults)
\usepackage{fancyhdr}
\pagestyle{fancy}
\lhead{}
\chead{}
\rhead{}
\lfoot{}
\cfoot{}
\rfoot{\thepage}
\renewcommand{\headrulewidth}{0pt}

\newcommand{\proofsep}{\vspace{-0.75em}}
\usepackage{booktabs}

\usepackage{graphicx}
% Improve math spacing around |,\left and \right
% See: https://tex.stackexchange.com/questions/2607/spacing-around-left-and-right/
\let\originalleft\left
\let\originalright\right
\renewcommand{\left}{\mathopen{}\mathclose\bgroup\originalleft}
\renewcommand{\right}{\aftergroup\egroup\originalright}

% Remove section numbering
\makeatletter
% we use \prefix@<level> only if it is defined
\renewcommand{\@seccntformat}[1]{%
    \ifcsname prefix@#1\endcsname
    \csname prefix@#1\endcsname
    \else
    \csname the#1\endcsname\quad
    \fi}
% define \prefix@section

%\newcommand\prefix@section{}
%\newcommand\prefix@subsection{}
%\newcommand\prefix@subsubsection{}
\makeatother

% Define new commands for "such that", the blackboard-bold letters and "closure".
\newcommand{\st}{\textit{ s.t.\ }}
\newcommand{\R}{\ensuremath{\mathbb{R}}}
\newcommand{\N}{\ensuremath{\mathbb{N}}}
\newcommand{\Z}{\ensuremath{\mathbb{Z}}}
\newcommand{\Q}{\ensuremath{\mathbb{Q}}}
\newcommand{\cl}{\ensuremath{\text{cl\hspace{0.1em}}}}

\newcommand{\card}{\ensuremath{\text{card\hspace{0.1em}}}}
\DeclareMathOperator{\blackboardE}{\mathbb{E}}
\newcommand{\expected}[1]{\blackboardE\left[#1\right]}
\newcommand{\expectedwhen}[2]{\blackboardE_{#2}\left[#1\right]}
\newlength{\postparenlength}
\setlength{\postparenlength}{0.14em}
\newcommand{\postparen}{\hspace{\postparenlength}}

\setlength{\parskip}{0.5\parindent}
\usepackage{url}
\usepackage{xcolor}

\usepackage[biblatex]{embrac}
\usepackage{siunitx}
\sisetup{
	group-separator={,},
	group-digits = integer,
	input-ignore = {,},
	input-decimal-markers = {.}
	}
\usepackage[authordate,isbn=false,backend=biber,autopunct=true,url=false]{biblatex-chicago}
%\DefineBibliographyExtras{american}{\stdpunctuation}
\renewcommand{\finalnamedelim}{\addspace \bibstring {and}\space}
%\renewcommand*{\bibfont}{\footnotesize}
%\DefineBibliographyStrings{english}{references = {References}}
%\setlength\bibitemsep{0pt}
\bibhang=\parindent
\addbibresource{./papers.bib}
\newcommand{\cex}{\textsc{cex}}
\newcommand{\apf}{\textsc{apf}}
\newcommand{\gpm}{\textsc{gpm}}
\newcommand{\msrp}{\textsc{msrp}}
\newcommand{\eitc}{\textsc{eitc}}
\newcommand{\gdp}{\textsc{gdp}}
\newcommand{\Var}{\text{Var}}

\usepackage{gitinfo2}  % requires git hooks!  See the Code/git_hooks folder in this repo.
\newcommand{\snippet}[1]{\input{./Generated_snippets/#1}\hspace{-0.15em}}
%\usepackage{footnote}
\usepackage{hyperref}
\hypersetup{colorlinks,
    linkcolor=black,
    filecolor=black,
    urlcolor=darkgray,
    citecolor=black,
    pdfpagemode=UseNone,
    pdftoolbar=false,
    pdftitle={Cars in Alaska},
    pdfauthor={Karl Dunkle Werner},
    pdfsubject={ARE second year paper},
    pdfcreator={},
    pdfproducer={},
    pdflang=en,
    unicode=true
}


\graphicspath{{./Plots/}, {./Plots/Daily/}}
\begin{document}
\thispagestyle{empty}
\setcounter{page}{0}
\vspace*{0.7in plus 0.3in minus 0.3in}

\begin{center}
    {\Huge Cars in Alaska}

    \textit{\Large Karl Dunkle Werner}

\vspace{1em}
  This draft: \gitCommitterDate{}
    % The \gitHash here will automatically update the URL as necessary
    \href{https://github.com/karldw/second_year_paper/tree/\gitHash}{
    (\textsc{\gitAbbrevHash{}})}\\
    \href{http://karldw.org/second_year_paper.pdf}{Click here for the most recent.}
\end{center}

\vspace{2in plus 1in minus 0.7in}

\begin{center}
    \begin{minipage}{0.7\linewidth}
        \begin{center}
            \textit{Abstract}
        \end{center}
        Abstract text here \ldots
    \end{minipage}
\end{center}

\pagebreak
%
% \textsc{\Large Outline}
%
% \begin{itemize}
%     \item Findings!
%     \begin{itemize}
%         \item Sales counts per capita
%         \item Sales volumes per capita
%         \item Mean efficiency (gallons per mile)
%         \item Vehicle value
%         \begin{itemize}
%             \item \msrp{}
%             \item Model year age
%             \item Auction price
%         \end{itemize}
%     \end{itemize}
%     \item Contribute to some literature strands:
%     \begin{itemize}
%         \item PIH
%         \item Fuel efficiency choice
%         \item IO of used auto markets
%     \end{itemize}
%     \item Papers to compare
%     \begin{itemize}
%         \item Hsieh
%         \item \textcite{Busse2015_weather_on_cars} about fickle car buying
%         \item Presentation about subsidy in EVs
%         \item \eitc{} literature (e.g.\ \cite{goodman2008eitc})
%     \end{itemize}
% \end{itemize}
%
% \pagebreak
\textsc{\Large Outline: draft for Feb 27}

\begin{itemize}
    \item Short paragraph on income effects
    \item Describe APF
    \item Contribute to some literature strands:
    \begin{itemize}
        \item PIH
        \item Fuel efficiency choice
        \item IO of used auto markets
    \end{itemize}
    \item Write theory of car buying, make predictions about results
    \begin{itemize}
        \item Explain why announcement date isn't going to be informative.
    \end{itemize}

    \item Findings -- small results
    \begin{itemize}
        \item Sales counts
        \begin{itemize}
            \item Very small coefficients, however you slice it.  Logs imply a somewhat noisier response.
            % anticipation_window_sale_count_weekly_notitle_coef_effect_mean.pdf
            % Then put these side-by-side:
            % variable_window_dd_sale_count_weekly_area_conf95_max_effect_mean.pdf
            % variable_window_dd_sale_count_log_weekly_area_conf95_max.pdf
        \end{itemize}

        \item Sales volumes
        \begin{itemize}
            \item  Only mention sales volumes in passing; the component pieces of sales\_count and sales\_pr are more interesting.
            \item The effects are small when estimated in levels, moderate with logs.
        \end{itemize}
        \item Mean efficiency (liters per 100km)
        \begin{itemize}
            % TODO in next iteration: look at mechanisms: if we control for vehicle age, what happens to the effect? What about weight?
            \item Overall, not a large effect. We can rule out any effect larger than about 10\% of the mean, and the coefficients aren't statistically significant.
            % anticipation_window_fuel_cons_weekly_notitle_coef_effect_mean.pdf
            % Then put these side-by-side:
            % variable_window_dd_sale_count_weekly_area_conf95_max_effect_mean.pdf
            % variable_window_dd_sale_count_log_weekly_area_conf95_max.pdf

        \end{itemize}
        \item Vehicle value, as measured by one of the following:
        \begin{itemize}
            \item \msrp{}
            \begin{itemize}
                % TODO: note that the last week of MSRP has a huge coefficient, but that says more about some year-end effect than this check.  I've dropped it to make the graph easier to see.
                \item  For \msrp{} the conclusions are dramatically different for levels and logs.
                    For levels, the estimated effects are insignificant, and the confidence intervals rule out large effects -- anything larger than 10 or 20\% of the mean.
                    % variable_window_dd_msrp_mean_weekly_area_conf95_max_effect_mean.pdf
                    For logs, the point estimates and confidence intervals allow for much larger results.
                    With point estimates around 0.35 and confidence interval upper bounds around 0.7 (for reasonable specifications; not those starting at the beginning), we can't reject large effects.
                    At the same time, the lower bound of the CIs is \emph{just} above zero, so I don't want to put too much stock in the statistical significance shown in the plot.
                    % variable_window_dd_msrp_mean_log_weekly_area_conf95_max.pdf
            \end{itemize}
            % \item Model-year age (TODO: add vehicle age in next draft)
            \item Auction price (valid as a unified measure of vehicle quality if there aren't market failures)
            \begin{itemize}
                \item  As with \textsc{msrp}, there's a difference between logs and levels here.  With levels, the results are insignificant, the point estimates are very close to zero and we can reject most results above 15\% of the mean price.  (The mean is \$14,912.13.)
                With logs, the point estimates again imply larger effects, around 0.3, with confidence intervals around 0.7.
                % Show two pairs here:
                % Pair a:
                % effects_individual_period_sales_pr_mean_weekly_coef_notitle.pdf
                % effects_individual_period_sales_pr_mean_log_weekly_coef_notitle.pdf
                % Pair b:
                % variable_window_dd_sales_pr_mean_weekly_area_conf95_max_effect_mean.pdf
                % variable_window_dd_sales_pr_mean_log_weekly_area_conf95_max.pdf
            \end{itemize}
        \end{itemize}
        \item Show that results are robust to dropping any particular year (make a graph)
        \item Acknowledge the issues of multiple testing -- they cut both ways, with my ``precise'' zeros being somewhat less precise.
    \end{itemize}


    \item Papers to compare
    \begin{itemize}
        \item Hsieh
        \item \textcite{Busse2015_weather_on_cars} about fickle car buying
        \item Presentation about subsidy in EVs
        \item \eitc{} literature (e.g.\ \cite{goodman2008eitc})
    \end{itemize}
\end{itemize}

\pagebreak
\setcounter{page}{1}
\section{Introduction}

% TODO short paragraph on the content why we might expect income effects.

The Alaska Permanent Fund is an account set up to hold and invest a portion of the state's mineral rights revenue.
These dividends are substantial, roughly \$1000--2000 per person, as shown in figure~\ref{fig:permanent-fund-payments-individual}.
The fund has been sending dividend payments in the fourth quarter of each year to Alaskan residents since the 1980s \parencite{hsieh2003}.
In recent years, the payment has happened on the first Thursday in October.
I will use this somewhat-arbitrary payment date as a source of variation in household's income, and therefore willingness to buy durable goods (specifically, cars).



It's worth mentioning a few of the dividend details before going on.
The dividend is paid out of investment earnings -- 10.5\% of the past five years' earnings -- not out of present-day oil revenue.
The Permanent Fund investments are made in a broad variety of stocks and bonds.%
\footnote{The asset allocation is detailed in the Alaska Permanent Fund Corporation's website, \url{http://apfc.org/home/Content/investments/assetAllocation2009.cfm}}
The dividend the same amount for all recipients, with no means testing or other adjustments.
Almost everyone in Alaska gets the dividend.
Major exceptions are people who have lived in the state for less than a year and people serving time for some crimes.
Approximately 91\% of the state's population applies and about 95\% of those applications are granted.

\begin{figure}[bht]
    \caption{\large Alaska Permanent Fund payments per person}
    % \vspace{-1.05em}
    \includegraphics[width=\linewidth]{permanent_fund_payments_individual_notitle.pdf}
    \label{fig:permanent-fund-payments-individual}

\noindent\textsc{Source:} Alaska Permanent Fund Division \parencite{apfd_payments_summary}.\par
		Figures are adjusted for inflation using the annual average \textsc{cpi} \parencite{fred_inflation}.
		The 2008 total includes a one-time, \$1200 bonus.
\end{figure}


The principal data source in my analysis is the Manheim wholesale used vehicle auction data.
The dataset records sale prices, vehicle identifiers, buyer and seller IDs and locations.
After various cleaning measures, described in section~\ref{sec:manheim-data-cleaning}, I have
\snippet{auctions_cleaned_total_obs_count.tex}
observations of vehicle sales.
These sales are between wholesalers such as auto dealerships, rental car agencies and vehicle leasing programs.
In most of the following analysis, I focus on wholesalers with billing zip codes in Alaska who buy vehicles in the auctions.
There are no auction sites in Alaska, so all of the purchases I observe for these dealers are from out of state.
While I see when the vehicles were bought at auction, I don't know when they were delivered.
Presumably, there can be some substantial delay shipping to Alaska; I examine the delay and anticipation effects as part of my analysis.
Additionally, these are \emph{used} vehicles only.
I won't be able to discuss the market for new vehicles beyond theoretical speculation.


Importantly, I don't observe final sales to customers, so I don't know the timing or their purchases or the prices they face.
I assume that cars bought by Alaskan dealers are sold to consumers with a not-to-long delay, since it's costly to hold unsold cars on the lot.
(I do exclude cars that are bought and resold within a year.)
The Manheim data are described in more detail in section~\ref{sec:manheim-data}.



\section{Model}


% TODO First explain in words, then make formal in math.  Figure out good, non-conflicting syntax.

\subsection{Consumers}


People have preferences for vehicle traits and vehicle ownership, but are constrained by their budgets.
They can therefore change what they buy when they have a check from the government.
% TODO: pull out one of the basic macro models of buffer-stock saving or something like it
Even though the check is anticipated, they don't fully smooth because they don't build up enough savings in the course of their normal consumption path.

There are also behavioral possibilities, both in lack of smoothing and people treating dividend checks differently than they would treat their own savings.
I won't really be able to disentangle these, since I don't have any situations where saved wealth is exogenously manipulated, but it's important to consider that responses may differ between this situation and others.

\subsection{Auto dealers}

``If the car has been there for more than three months, the dealer will be more anxious to sell it as quickly as possible.''%
\footnote{\textcite{usnews_car_deals}} %% --elaborate here...

To make the model less complex, I'll assume auto-dealers don't exercise market power when they buy or sell their used cars, that they have a deep understanding of consumers' preferences, and that it's costly to hold inventory.

Assuming consumers buy more used cars upon receiving their dividend checks, dealers adjust.
Specifically, the dealers know what kinds of cars will be popular and will buy these beforehand, making sure they're available when the dividends arrive.

It's also possible the dividend checks influence \emph{new} car purchases.
I don't have data on these.
%TODO: should I model the unobserved new car purchases as well, having dealers maximize over both new and used sales?

Assuming away market power is clearly a big assumption\ldots
% TODO: any justification? What would the implications be if they had monopoly power in selling to consumers or monopsony power in buying in auctions.
% TODO: how are the auctions designed? Sealed bid? English? Some of them have reserves...

% TODO: Another story I could tell is one of price effects.  If Alaskan dealers pay more for the same VIN10 in years with high dividends, we might think there's something going on.
% To get that right, it's important to control for vehicle quality carefully; a higher price could indicate market shifts, or just that Alaskan dealers are shifting to newer used cars.
% (Both are interesting, but they're different stories.)

\section{Data}
%\label{sec:data}

\subsection{Wholesale auto auctions}
\label{sec:manheim-data}

% TODO: Repeat a little about the used car data and the things I can observe. Make clear again that I can't see sales to final consumers.

\snippet{clean_car_auctions_resold.tex}
\subsubsection{Cleaning auction data}
\label{sec:manheim-data-cleaning}

\begin{table}[hbt]
    % TODO: make "table 1" serif
    \caption{Cleaning Manheim auction data}
    \label{tab:cleaning_manheim}
    % TODO: Make the numbers here better by having parse-numbers = true (currently broken because the snippet gets added after siunitx tries to parse the number)
    \sisetup{parse-numbers = false}
\begin{tabular}{l S p{0.493\linewidth}}
    \toprule
	Elimination category & \multicolumn{1}{l}{Count removed} & Details\\
	\midrule

    % Bad sale date & \snippet{load_manheim_filter_bad_date_count.tex} & Reported sale date isn't a date.\\
	% \addlinespace

    Duplicate sales & \snippet{clean_car_auctions_resold.tex} & Same \textsc{vin} sold twice within a year.\\
    \addlinespace
    Weird vehicles & \snippet{clean_car_auctions_weird_vehicles.tex} & Trailers, boats, air compressors, golf carts, vehicles with incomplete bodies, \textsc{atv}s and \textsc{rv}s.\\
	\addlinespace
    Bad odometer & \snippet{clean_car_auctions_bad_odometer.tex} & Auction comments indicate odometer is flawed.\\
	\addlinespace
    Damaged & \snippet{clean_car_auctions_damaged.tex} & Auction comments indicate vehicle is damaged. \\
	\addlinespace
    Bad price & \snippet{clean_car_auctions_price.tex} & Auction price seems unreasonable, outside the interval $[100, \min\{80000, 1.5\times \textsc{msrp} \}]$.\\
	\addlinespace
    Canadian & \snippet{clean_car_auctions_canadian.tex} & Auction comments indicate vehicle is Canadian.\\
    \bottomrule
    \addlinespace
\end{tabular}
\footnotesize
\textsc{Notes:} Data are from US Manheim auctions, 2002--2014.
There are \snippet{auctions_uncleaned_total_obs_count.tex} sales before cleaning and \snippet{auctions_cleaned_total_obs_count.tex} sales after.
\snippet{auctions_cleaned_alaska_obs_count.tex} of the cleaned sales are to Alaskan buyers.
Cars that are listed for auction but not sold are not included.
Rows are removed in the order listed.

\end{table}

The Manheim data are voluminous, and they include many kinds of sales outside the scope of this research.

\subsection{VIN decoder}
\subsection{Gas prices}
\subsection{CEX}
\subsection{Vehicle registrations}


\section[DD Results]{Difference-in-differences results}

For each of the outcomes below:
\begin{itemize}
    \item Make a plot of the time trend of Alaska vs other states.
\end{itemize}


\subsection{Picking control states}
In considering which states are good controls for Alaska, it's worth diving a little more into the structure of the Manheim auction data.
Auctions occur in 32 states and Puerto Rico, and Alaska is not among them.
Therefore all purchases by Alaskan buyers are at out-of-state auctions.
For the simple difference-in-differences below, I've chosen Idaho, Oregon and Utah.
The choice is somewhat subjective; these states have a large fraction of their out-of-state purchases in the auction states where Alaskan dealers are buying.
This heuristic is appealing because it captures the markets where Alaska is operating, as well as additional factors from buying out of state.
% See the code in find_match_states_crude_unmemoized for full details..
A more robust approach is to use best-subset or synthetic controls methods described in \textcite{DoudchenkoImbens2016DD}.


\subsection{Picking an anticipation window}


Anecdotally, dealers know that customers buy more cars around the dividend date and around tax refunds, but they don't make additional efforts to stock up on vehicles before the dividend is sent.
Instead, they just try to keep a certain number of cars on their lot, buying more as necessary.

\begin{figure}[hbt]
    \caption{Anticipation effects for sales count}
    \label{fig:anticipation_window_sale_count}
    \includegraphics[width=\linewidth]{anticipation_window_sale_count_weekly_notitle_coef_effect_mean.pdf}

    {\footnotesize
    The plot shows the Alaska $\times$ anticipation coefficients ($\beta_2$) from the regression in equation~\ref{eqn:cars_sold_with_anticipation}, with varying anticipation windows.
    Error bars indicate 95\% confidence intervals.
    %The horizontal line at \snippet{sales_count_weekly_std_dev.tex} is the standard deviation of weekly counts.
    All windows end at $-1$ week.
    }

\end{figure}
\begin{figure}[hbt]
     \caption{Anticipation effects for sales totals}
     \label{fig:anticipation_window_sale_tot}
        \includegraphics[width=\linewidth]{anticipation_window_sale_tot_weekly_notitle_coef_effect_mean.pdf}

    {\footnotesize
    The plot shows the Alaska $\times$ anticipation coefficients ($\gamma_2$) from the regression in equation~\ref{eqn:cars_sold_with_anticipation}, with varying anticipation windows.
    Error bars indicate 95\% confidence intervals.
    %The horizontal line at \snippet{sales_tot_weekly_thousands_std_dev.tex} is the standard deviation of weekly sales volume (in thousands).
    All dollar amounts are in thousands of 2016 dollars.
    All windows end at $-1$ week.
    }
\end{figure}

% TODO: Make the event-study plot Hal was talking about; for my zero to be credible, these need to *remain* pretty parallel.

Anecdata are often wrong, so maybe it's worth considering the theoretical predicting a dealer anticipation effect.
Theory doesn't tell us, however, how long that anticipation window should be.
To be as flexible as possible, I have estimated the results for varying windows from one to nine weeks of anticipation, as shown in figures~\ref{fig:anticipation_window_sale_count} and~\ref{fig:anticipation_window_sale_tot}.
In all cases, the anticipation window ends the immediately before the dividend day.

None of the estimates are statistically significant, and they are all far below the standard deviation of weekly sales counts and sales totals, as indicated by the horizontal lines at the top of the graphs.
(Different states have somewhat different standard deviations, the pooled standard deviation is not driven by states with particularly large variation.
The 95\% confidence interval of the anticipation estimates less than are standard deviation for Alaska and every control state, as demonstrated in figure~\ref{bonusfig:anticipation_window_sale_count_and_tot_weekly_states_sd}.)
I default to calculating the pooled standard deviation across Alaska and the control states and across all weeks in the data.
Calculating the standard deviation on only periods within the event window -- from 70~days before the dividend is sent to 70~days after -- gives slightly larger standard deviations, and therefore even smaller effect sizes.

Using daily data produces similar results.
The coefficients are somewhat more variable and a couple are statistically significant, but they're still much smaller than the standard deviation.
% (See figure~\ref{bonusfig:anticipation_window_sale_count_daily}.)


\begin{figure}[hbt]
     \caption{Sales totals effect size across windows, max 95\% CI}
     \label{fig:variable_window_sale_tot_conf95_max_effect}
        \includegraphics[width=\linewidth]{variable_window_dd_sale_tot_weekly_area_conf95_max_effect.pdf}

    {\footnotesize
    % TODO: write an equation for this, put in the appropriate greek letter. (use \psi or change the math below)
    \textsc{Notes:} The plot shows the Alaska $\times$ anticipation coefficients on total sales from a regression with a varying ``anticipation'' window.
    The $y$-axis shows is the start of the window relative to the dividend day, while the $x$-axis codes the end of the window.
    The size of the points is the maximum of the absolute value of the effect size of the 95\% confidence intervals: $\max \big\{ \hat{\psi} - 1.96 \sqrt{\Var(\hat{\psi})},$ $ \hat{\psi} + 1.96 \sqrt{\Var(\hat{\psi})} \big\}$.
    A standard DD regression -- where the window begins at zero and ends at the end of the period -- is the rightmost point on the $y=0$ line.
    }
\end{figure}

\begin{figure}[hbt]
     \caption{Maximum sales count effect size confidence intervals}
     \label{fig:variable_window_sale_count_conf95_max_effect}
        \includegraphics[width=\linewidth]{variable_window_dd_sale_count_weekly_area_conf95_max_effect.pdf}

    {\footnotesize
    % TODO: write an equation for this, put in the appropriate greek letter. (use \psi or change the math below)
    \textsc{Notes:} The plot shows the Alaska $\times$ anticipation coefficients on sale count from a regression with a varying ``anticipation'' window.
    The $y$-axis shows is the start of the window relative to the dividend day, while the $x$-axis codes the end of the window.
    The size of the points is the maximum of the absolute value of the effect size of the 95\% confidence intervals: $\max \big\{ \hat{\psi} - 1.96 \sqrt{\Var(\hat{\psi})},$ $ \hat{\psi} + 1.96 \sqrt{\Var(\hat{\psi})} \big\}$.
    A standard DD regression -- where the window begins at zero and ends at the end of the period -- is the rightmost point on the $y=0$ line.
    }
\end{figure}

On the other hand, one might put more credence in the anecdotes.
To show that there is still no effect on vehicle sales, I'll estimate an even more flexible treatment window.




% TODO: in addition to comparing vs the std dev, also compare as a pct of the mean.

% \begin{figure}[hbt]
%     \caption{Treatment coefficients on Alaska $\times$ anticipation: total sales}
%     \label{fig:anticipation_window_sale_tot}
%     \includegraphics[width=\linewidth]{anticipation_window_sale_tot_weekly_notitle_pooled_sd.pdf}
%
%     {\footnotesize
%     The plot shows the Alaska $\times$ anticipation coefficients ($\gamma_2$) from the regression in equation~\ref{eqn:cars_sold_with_anticipation}, with varying anticipation windows.
%     Error bars indicate 95\% confidence intervals.
%     The horizontal line at \$\snippet{sales_tot_weekly_thousands_std_dev.tex} is the standard deviation of weekly sales volume (in thousands).
%     All windows end at $-1$ week.
%     }
% \end{figure}

\subsection{Volume of cars sold}
% TODO: move this section up so I can talk about number and volumes of cars sold
\begin{align}
    \label{eqn:cars_sold_with_anticipation}
    \text{Cars sold} = & \ \beta_1 \ \text{anticipation} + \beta_2 \ \text{Alaska} \times \text{anticipation} + \beta_3 \ \text{post-dividend} \\ \nonumber
    &+ \beta_4 \ \text{Alaska} \times \text{post-dividend} + \text{state and year fixed effects}
    %\\
    % \text{Sale volume} = & \ \gamma_1 \ \text{anticipation} + \gamma_2 \ \text{Alaska} \times \text{anticipation} + \gamma_3 \ \text{post-dividend} \\ \nonumber
    % &+ \gamma_4 \ \text{Alaska} \times \text{post-dividend} + \text{state and year fixed effects}
\end{align}

Collapse data to a state-by-day panel and run DD regs of sale count per capita and sale volume per capita.
Use all years, use event time relative to the dividend day and don't take into account any anticipation.
Control for either a year FE or state \gdp{}. Control for either daily or day-of-week FE.
As a default, the cars sold in the set of states where Alaskan buyers are active (in a given year).
Force the panel to be balanced by adding zeros (if necessary).

This is theoretically ambiguous -- sales of \emph{used} cars could go up or down with an income shock, or be unchanged.





\subsection{Efficiency of cars sold}
Repeat the previous, but with the average \gpm{} of vehicles sold.

\subsection{Quality of cars sold}
Repeat the previous, but with the average \msrp{} of vehicles sold.

Repeat the previous, but with the model-year-age (relative to auction date) of vehicles sold.


\section{Synthetic results}
Do the previous DD section again, but with synthetic controls or, maybe, the fancier frameworks of
\textcite{DoudchenkoImbens2016DD} or \textcite{Xu2016}




% \section{Replicating Hsieh}
% % TODO: discuss his estimation setup
% % TODO: estimate his setup
% % TODO: answer his question with a DD framework instead
%
% \subsection{All expenses}
% \subsection{Cars only}
%
% There are very few car sales in the CEX data,
% % TODO: how many?





\pagebreak
% References
\printbibliography



\appendix

\section{Bonus figures}

\begin{figure}[hbt]
    \caption{Treatment coefficients on Alaska $\times$ anticipation}
    \label{bonusfig:anticipation_window_sale_count_daily}
    \includegraphics[width=\linewidth]{anticipation_window_sale_count_daily_notitle.pdf}
    {\footnotesize
    The plot shows the Alaska $\times$ anticipation coefficients ($\beta_2$) from the regression in equation~\ref{eqn:cars_sold_with_anticipation}, with varying anticipation windows.
    All windows end at $-1$ event day.
    }
\end{figure}

\begin{figure}[hbt]
	\caption{Treatment coefficients on Alaska $\times$ anticipation, with state-specific standard deviation lines}
	\label{bonusfig:anticipation_window_sale_count_and_tot_weekly_states_sd}
	\includegraphics[width=\linewidth]{anticipation_window_sale_count_weekly_notitle_states_sd.pdf}

	\includegraphics[width=\linewidth]{anticipation_window_sale_tot_weekly_notitle_states_sd.pdf}
\end{figure}

\begin{figure}[hbt]
	\caption{Alaska-only and pooled mean and standard deviations}
	\label{bonusfig:alaska_vs_pooled_mean_sd}
	\includegraphics[width=\linewidth]{comparison_alaska_vs_pooled_mean_sd.pdf}
\end{figure}


% \begin{figure}[hbt]
%     \caption{Treatment coefficients on Alaska $\times$ anticipation}
%     \label{bonusfig:anticipation_window_sale_count_daily}
%     \includegraphics{anticipation_window_sale_count_daily_notitle.pdf}
%     {\footnotesize
%     The plot shows the Alaska $\times$ anticipation coefficients ($\beta_2$) from the regression in equation~\ref{eqn:cars_sold_with_anticipation}, with varying anticipation windows.
%     All windows end at $-1$ event day.
%     }
% \end{figure}

\end{document}
