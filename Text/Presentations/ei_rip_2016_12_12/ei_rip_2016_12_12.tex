\documentclass[aspectratio=169]{beamer}
\usepackage[utf8]{inputenc}
\usepackage[T1]{fontenc}
\usepackage{amssymb,amsmath,amsthm}
%\usepackage{enumitem}
\usepackage[osf]{libertine}
%\usepackage{carlito}
\usepackage{inconsolata}
%\usepackage{sectsty}

%\allsectionsfont{\sffamily\mdseries}
\usepackage[activate={true,nocompatibility},final,tracking=true,spacing=true]{microtype}
\microtypecontext{spacing=nonfrench}
% Set color theme

\newcommand{\proofsep}{\vspace{-0.75em}}
\usepackage{booktabs}

\usepackage{graphicx}
% Improve math spacing around |,\left and \right
% See: https://tex.stackexchange.com/questions/2607/spacing-around-left-and-right/
\let\originalleft\left
\let\originalright\right
\renewcommand{\left}{\mathopen{}\mathclose\bgroup\originalleft}
\renewcommand{\right}{\aftergroup\egroup\originalright}

% Remove section numbering
\makeatletter
% we use \prefix@<level> only if it is defined
\renewcommand{\@seccntformat}[1]{%
	\ifcsname prefix@#1\endcsname
	\csname prefix@#1\endcsname
	\else
	\csname the#1\endcsname\quad
	\fi}
% define \prefix@section
\newcommand\prefix@section{}
\newcommand\prefix@subsection{}
\newcommand\prefix@subsubsection{}
\makeatother

% Define new commands for "such that", the blackboard-bold letters and "closure".
\newcommand{\st}{\textit{ s.t.\ }}
\newcommand{\R}{\ensuremath{\mathbb{R}}}
\newcommand{\N}{\ensuremath{\mathbb{N}}}
\newcommand{\Z}{\ensuremath{\mathbb{Z}}}
\newcommand{\Q}{\ensuremath{\mathbb{Q}}}
\newcommand{\cl}{\ensuremath{\text{cl\hspace{0.1em}}}}
\newcommand{\card}{\ensuremath{\text{card\hspace{0.1em}}}}

\DeclareMathOperator{\blackboardE}{\mathbb{E}}
\newcommand{\expected}[1]{\blackboardE\left[#1\right]}
\newcommand{\expectedwhen}[2]{\blackboardE_{#2}\left[#1\right]}
\newlength{\postparenlength}
\setlength{\postparenlength}{0.14em}
\newcommand{\postparen}{\hspace{\postparenlength}}
\usepackage{url}
\usepackage{xcolor}

\usepackage{siunitx}
\sisetup{
	group-separator={,},
	group-digits = integer,
	input-ignore = {,},
	input-decimal-markers = {.}
}
\usepackage[biblatex]{embrac}

\usepackage[authordate,isbn=false,backend=biber,autopunct=true]{biblatex-chicago}

\addbibresource{refs.bib}


\usetheme{Szeged}
\usecolortheme{dove}
\beamertemplatenavigationsymbolsempty
\title[Cars in Alaska]{\textsc{EI Research in Progress\\Alaska Permanent Fund}}
\author{Karl Dunkle Werner}
\institute{}
\date{December 12, 2016}
%\usefonttheme{structuresmallcapsserif}
\usefonttheme{serif}


\usepackage[overlay,absolute]{textpos}
\setlength{\TPHorizModule}{\textwidth}
\setlength{\TPVertModule}{\textwidth}

%\usepackage{footnote}
\usepackage{hyperref}
\hypersetup{colorlinks,
	linkcolor=darkgray,
	filecolor=black,
	urlcolor=gray,
	citecolor=darkgray,
	pdfpagemode=UseNone,
	pdftoolbar=false,
	pdftitle={Cohort Presentation -- Alaska Permanent Fund},
	pdfauthor={Karl Dunkle Werner},
	pdfsubject={},
	pdfcreator={},
	pdfproducer={},
	pdflang=en,
	unicode=true
}

%\graphicspath{.}
\begin{document}
% {
% \usebackgroundtemplate{\includegraphics[width=\paperwidth,height=\paperheight]{./boston_leaks.png}}%
\begin{frame}[plain]
	\maketitle
	% \begin{textblock}{0.6}(0.5,0.2)
	% 	\begin{center}
	% 		{\Huge Hausman and \par \vspace*{0.15em}Muehlenbachs} {\LARGE (2016)} \par
	% 		\vspace*{0.5em}
	% 		Discussion by Karl Dunkle Werner\par
	% 		\textsc{are} 261,
	% 		September 21, 2016
	% 			\end{center}
	% \end{textblock}

	% \begin{textblock}{1}(0.001,0.61)
	% 	\footnotesize \textcolor{gray}{Source: \href{https://www.edf.org/climate/methanemaps/city-snapshots/boston}{Environmental Defense Fund}}
	% \end{textblock}
\end{frame}
% }
% Outline page
%\begin{frame}{Outline}
%	\tableofcontents
%\end{frame}


\begin{frame}{To do}
	\begin{itemize}
		\item Add a photo of Alaska, development style
		\item Add some aggregate trend plots
		\item Discuss how the dividend check works:
		\begin{itemize}
			\item Timing of receipt
			\item Timing of announcement
			\item Payment amount (fixed formula from invested funds, except 2008 and 2016)
		\end{itemize}
	\end{itemize}
\end{frame}
\begin{frame}{Outline}
	\begin{itemize}
		\item Cars matter for energy:
		\begin{itemize}
			\item They're how we consume gasoline
			\item Energy efficiency gap
			\begin{itemize}
				\item Does relaxing income constraint matter for the composition of cars people buy?
				\item Do people treat money from this kind of dividend differently than money from work?
			\end{itemize}
		\end{itemize}
		\item And for other questions:
		\begin{itemize}
			\item Cars are expensive, and important from a pure household economics perspective.
			\item The APF is a relatively rare example of a government taking resource wealth and refunding it directly to citizens -- interesting to see the effects.
			\item Also, from a non-energy perspective, the strongest form of the PIH says that we should see no effect in composition or volume of car sales -- maybe interesting. (People have previously used the APF to support the PIH, see \cite{hsieh2003})
		\end{itemize}

		\item Acknowledge weakness:
		\begin{itemize}
			\item Not the best identification
			\item The variation I really want, dose--response in the amount of the APF payment, can't be identified separately from macro trends.
			\item Only have wholesale car data, which is great for some things and terrible for others.
		\end{itemize}
	\end{itemize}
\end{frame}


\section{Background}

{
\usebackgroundtemplate{\includegraphics[width=\paperwidth,height=\paperheight]{../../Plots/permanent_fund_payments_individual.pdf}}
\begin{frame}[plain]
\end{frame}
}

{
\usebackgroundtemplate{\includegraphics[width=\paperwidth,height=\paperheight]{../../Plots/permanent_fund_payments_aggregate.pdf}}
\begin{frame}[plain]
\end{frame}
}


\begin{frame}{Hsieh (2003)}
	\[
	\log \left(\frac{ C_h^{\textsc{q4}} }{ C_h^{\textsc{q3}} } \right) = \alpha_1
	\frac{ \textit{\textsc{pfd}}_t \times \textit{Family size}_h }{\textit{Family income}_h} + \mathbf{z}_h' \mathbf{\alpha}_2
	\]

	\begin{itemize}
		\item Do people smooth their consumption when they get the payment?
		\begin{itemize}
			\item Measured by log of the ratio of \textsc{q4} to \textsc{q3} consumption.
		\end{itemize}
		\item Use differences in \textsc{pfd} payout and family size as variation in amount household receives in last quarter of the year.
		\begin{itemize}
			\item Can't control for both year fixed effects and family size.
		\end{itemize}
		%\item Also compares Alaska vs.\ rest of US
	\end{itemize}

\end{frame}

\section{Outcomes}
\begin{frame}{Big expenses}
	Data I have:
	\begin{itemize}
		\item County-by-quarter counts of new vehicle registrations
		\item Wholesale auto auctions, with buyer's and seller's billing zip code
		\item Quarterly consumer expenditure data (\textsc{cex})
	\end{itemize}

	Data I want:
	\begin{itemize}
		\item Medical expenditures (state-by-day or state-by-week)
		\item Debt info?
		\item Other stuff?
	\end{itemize}
\end{frame}


\section{Methods}

\begin{frame}{Difference in Differences}
	Still lots of cleaning to do\ldots
\end{frame}

{
\usebackgroundtemplate{\includegraphics[width=\paperwidth,height=\paperheight]{../../Plots/auctions_2004_alaska_vs_other.pdf}}
\begin{frame}[plain]
\end{frame}
}

\begin{frame}{Synthetic Controls!}
\end{frame}


\begin{frame}{Generalized Synthetic Controls!!}
	Synthetic controls, with time-varying factors.

	Depends on $N\to\infty$ and $T\to\infty$.
\end{frame}

\begin{frame}{Xu (2016): election-day registration example}
	\vspace*{-0.1cm}
	\begin{center}
		\includegraphics[page=25, trim = 2.5cm 15.5cm 2.0cm 3.0cm, clip,height=7cm]{../../../Papers/Econometrics/Xu_2016.pdf}
	\end{center}
\end{frame}



\end{document}
